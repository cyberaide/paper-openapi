\hyphenation{al-though Al-though}

\newcommand{\Cloudmesh}{%
$\blacksquare\hspace{-0.1cm}
\blacksquare\hspace{-0.1cm}
\blacksquare\hspace{-0.1cm}
\blacksquare\hspace{-0.1cm}
\blacksquare\hspace{-0.1cm}
\blacksquare\hspace{-0.1cm}
\blacksquare\hspace{-0.1cm}
\blacksquare\hspace{-0.1cm}
\blacksquare\hspace{-0.1cm}
$ ~}

\newcommand{\COMMENT}[1]{}
\newcommand{\CENSOR}[2]{#2~}

\newcommand{\BLOCKCENSOR}[2]{

\bigskip
\fbox{
\begin{minipage}{0.8\columnwidth}
{\bf\em #1
}
\end{minipage}
}}


\section{Introduction}

In today's application, scientists want to share their services with
many colleagues while not only offering the services as bare metal
programs but exposing the functionality as a Software as a Service
(SaaS). This has the advantage that the services can be readily reused
by other applications and hosted in the cloud, allowing access to
state-of-the-art services including compute and data that otherwise would
not be accessible to individual domain experts. Through the increased
availability, resource constraints can be reduced, and scientists can
we listoffer their analytics workflows as services to the community. This may
include long-lasting services enabled by cloud computing as part of
its Software as a Service (SaaS) paradigm or for smaller analytics
functions as microservices. Furthermore, a subset of analytics
functions can be offered as part of a serverless computing model,
elevating the penetration from a pure bare metal solution to a
multi-pronged cloud-based service offering.

While working with many professionals, researchers, and students, we
found that the barriers to entry to accomplish this goal remain very
high, and would elude many domain experts using them as they have neither the
expertise nor the time to learn the expertise to conduct the
infrastructure-related tasks integrating DevOps and analytics
tasks. Although recent developments, especially on the serverless
computing side, have made progress, we ought to leverage the existing
expertise of the domain scientists while automating the creation of
various services to provide SaaS, microservices, and serverless
computing automatically.

Having worked with the community, we found that the educational steps
involved for a beginner take about two to four months to get up to a
level where the development of cloud-based services is possible. We
set the goal to explore if it is possible to drastically reduce the
time needed to create such services.

For this reason, we developed a sophisticated but easy to use
framework that takes a regular Python function and converts it
automatically into a secure REST service and OpenAPI
specification \cite{openapi} that can be reused in the ecosystem of
cloud services. We used this framework to create different AI-based
REST services to showcase the approach's validity. In this paper we
discuss examples taken from from SciKit-learn \cite{scikit-learn} and
benchmark the execution of the resulting REST services on various
clouds and an IoT device.

The paper is structured as follows. In \Section{sec:background} we
start with a brief background section to allow domain
experts to catch up with the terminology and concepts used in our
architecture. The background analysis leads us to our requirements
presented in \Section{sec:requirements} and our architectural design
shown in \Section{sec:architecture}. Our benchmarks are collected
in \Section{sec:benchmark}. We present our conclusion
in \Section{sec:conclusion}.

\begin{comment}

In the appendix, a small number of useful notes are provided to ease
replication of what we have achieved by others. In the final
publication, the appendix can be removed with a link to our manual for
the pilot framework presented here \cite{cloudmesh-manual,
cloudmesh-openapi} where we will include the content of the appendix.

\end{comment}

\section{Background}
\label{sec:background}

We provide a small summary of activities that motivated this research and so
that domain experts can get a small introduction to concepts that we
use to implement our architecture. It is beyond the scope of this
paper to give more detailed introductions in topics such as IaaS,
SaaS, microservices, serverless computing, OpenAPI, and REST
services. The sections will, however, be useful as a starting point
for further research to the reader.

\subsection{The Big Data Reference Architecture}

NIST has developed a Big Data Reference Architecture as part of the
NIST Big Data Interoperability Framework (NBDIF)\cite{nist-v6} and
identified several use cases that motivate that
architecture \cite{nist-v3}. The architecture is depicted
in \Figure{fig:bdra} and includes the following components: Data
Provider, Big Data Application Provider, Big Data Framework Provider,
Data Consumer and System Orchestrator as well as two overarching
fabrics: security and privacy and system management. There are three
types of linkages, namely \emph{Big Data Information
flow}, \emph{Service Use and Software Tools}, and \emph{algorithms
transfer}. The architecture presents a level of abstraction to define
Big Data applications. Components that implement sophisticated
functionality work in concert to address the challenging creation of
instantiating architectures beyond the conceptual stage. As such, the
components interact with each other that are expressed through the
linkages within the NBDIF. Although an extensive set of documentation
exists a reference implementatio has not been developed by NIST
~\cite{nist-v1,nist-v2,nist-v3,nist-v4,nist-v5,nist-v6,nist-v7,nist-v8,nist-v9}}.

\begin{figure}[htb]
\centering

\includegraphics[width=1.0\columnwidth]{images/NIST-RA-latest-crop.pdf}

\caption{NIST Big Data Reference Architecture \cite{nist-v8}}

\label{fig:bdra}
\end{figure}

\COMMENT{Volume 8 is of especial importance as it provides a
blueprint for interfaces to address the needs of a generalized Big Data
Architecture as summarized in~\cite{cloudmesh-openapi}\cite{las20book-cloudeng}.}

In contrast to the NIST effort, our
effort builds the theoretical basis of needed activities while
expanding it to cloud providers and services focusing on {\em
Analytics Services}. 

In a previous effort, we have developed a reference implementation
that follows the architecture laid out in \censor{NBDIF} and is easy
to use by scientists.  However, it focused mostly on multi-cloud
provider access via REST services and command-line tools. \COMMENT{The
reference implementation is done as part of the \Cloudmesh project,
which was one of the first hybrid multi-cloud provider
interfaces.}

\begin{comment}
even including cloud technologies \censor{that are no
longer active,} such as Eucalyptus \cite{www-eucalyptus},
OpenCirrus \cite{opencirrus}, FutureGrid \cite{futuregrid}, and Comet
Cloud \cite{las-comet}}.
\end{comment}

It supports clouds such as
AWS \cite{www-aws}, Azure \cite{www-azure}, Google Cloud
Platform \cite{www-google}, Oracle \cite{www-oracle-cloud}, and
OpenStack \cite{www-openStack}. It can also offer furtheradditinal value as its goal is also to 
explore the integration of
MapReduce frameworks such as
Hadoop \cite{www-hadoop} and Spark \cite{www-spark}, as well as
container-based frameworks such as Docker \cite{www-docker}, and
Kubernetes \cite{www-kubernetes}.

However, the work presented here focuses not on the interaction with low-level
cloud provider interfaces, but on the {\em creation of high-level analytics services
that can be automatically created} and hosted on any of the
clouds. \COMMENT{supported by \Cloudmesh.} This is a non-trivial
effort due to the large number of technologies involved and is outside
of the expertise of domain scientists. However, the use of \Cloudmesh
makes it possible for the domain scientist to easily access these
services and leverage our more than a decade long experience in this
field.

\CENSOR{The previous work provides us with a blueprint on how to proceed.  We
list the following main findings of our earlier work that we leverage
as part of this work.}{The architectural decissions of our effforts are defined
by the following observations that will be integrated in our requirements:}

\begin{description}
  
\item[Software Defined Analytics Services and Applications.] ~\\ 
  Just as idntified by \censor{NBDIF}, the utilization of \emph{DevOps} to
  deliver Software-Defined (SD) Big Data applications is of utmost
  importance for the design of reusable services and
  components \censor{~\cite{cloudmesh-manual}}.
  
\item[Multi-cloud Provider Interfaces.]
\BLOCKCENSOR{A section was removed to fulfull double blind review conditions.}

\COMMENT{Volume 8 was through community
  input shaped in such a form that it allows multi-cloud
  interfaces. Such interfaces have been in practical use in our
  software and showcase the validity of the NIST-BDRA approach.}
  
  It is
  clear that we need to introduce such multi-cloud and multi-service
  interfaces for analytics-related tasks whenever possible as
  motivated in our introduction.

\item[Use Case Collection.] 
\CENSOR{\item[Use Case Collection.] NIST has provided as part of the NIST BDRA
  document Vol. 6~\cite{nist-v6} several use cases that can be
  analyzed and from which common big data services can be
  detected. These use cases were sufficient to drive the NIST BDRA
  document \cite{nist-v6} and allowed the community to investigate
  initial implementations. These use cases also motivate the work
  conducted in this effort.}{To validate the work we need to identify
  a number of simple use cases such as provided by SciKit Learn. In
  other efforts we work on integrated usecases dealing with industry
  as well as High Performance Computing Research use cases leveraging
  AI service generation.}


\item[Independent API Specification Leveraging OpenAPI.]
  \CENSOR{~\\ Although the use of OpenAPI \cite{openapi,openapi-tools}
  is not required as part of the NIST specification, it can be used to
  formulate services in a language-independent fashion. Hence, it
  allows {\em creating, evolving and promoting a vendor-neutral
  description format}. This is important to provide for our analytics
  services approach to promote a vendor-neutral and independent
  effort.}  {The use of OpenAPI \cite{openapi,openapi-tools} allows
  {\em creating, evolving and promoting a vendor-neutral description
  format}.  Although beneficial for interoperability concerns, it is
  too complex, so a simpler approach is needed.}


\item[API's and Tools Targeting A Multi-Layered Architecture.] The
  community as identified that support for tools, services, and APIs
  on multiple levels in a multi-layered architecture is needed. While
  some users expect a generalized specification other users may
  require access on the command line, deployed services, or even a
  Jupyter notebook.
  
\item[Lower the Entry Barrier.] We observe that in many cases, the entry-level to
  define API specification is too high for many. This is the case for
  domain experts in the analytics community that often lack the
  necessary expertise for general service integration and deployment.

\end{description}

\COMMENT{previous work provides us with a blueprint on how to proceed,
which we summarize as follows:}

Hence, our goal can be summrized as follows:

{\em Develop an easy to use framework that allows the scientists (a)
to develop shareable analytics components (b) allow for the deployment
of them, and (c) allow for the easy reuse of the services by community
members leveraging the deployments.}


\begin{comment}
\subsection{REST}\label{rest}

One of the most common architectural styles for cloud-related services
is based on {\bf R}epr{\bf E}sentational {\bf S}tate {\bf T}ransfer
(REST). REST often uses the HTTP protocol for the CRUD functions,
which create, read, update, and delete resources. It is important to
note that REST is not a standard, but it is a software architectural
style for building network services.  When referred to as a part of
the HTTP protocol, REST has the methods of GET, PUT, POST, and
DELETE. These methods are used to implement the CRUD functions on
collections and items for which REST introduces abstractions for
managing these collections and single resources \cite{las-book-cloud}
as explained in \Figure{fig:rest}.


\begin{figure}[htb] 

\begin{adjustbox}{minipage=0.9\columnwidth,
                  margin=10pt 5pt,%
                  bgcolor={black!2},%
                  frame=0.01pt}%
{\footnotesize
\begin{description}

\item[Collection of resources.] Assume the URI,  \verb|http://.../resources/|, identifies a
  collection of resources. The following CRUD functions would be
  implemented:

  \begin{description}
  \item
    [GET:] List the URIs and details about the collection's
    items.
  \item[PUT:] Replace the collection with a different collection.
  \item[POST:] Make a new entry in the collection. The operation
    returns new entry's URI and assigns it automatically.
  \item
    [DELETE:] Delete the collection.
  \end{description}
  \bigskip

\item[Single Resource.] Assume the URI, \verb|http://.../resources/item1|, identifies a
  single resource in a collection. The following CRUD functions would be
  implemented:

  \begin{description}
  \item
    [GET:] Fetch a representation of the item in the collection,
    extracted in the appropriate media type.
  \item
    [PUT:] Replace the item in the collection. If the item does
    not exist, then create the item.
  \item
    [POST:] Typically, not used. Treat the item as a collection
    and make a new entry in it.
  \item
    [DELETE:] Delete the item in the collection.
  \end{description}
\end{description}
}
\end{adjustbox}
\caption{REST definitions for a collection and single resources.}
\label{fig:rest}
\end{figure}

Because REST has a defined structure, there are tools that manage
programming to REST style architectures. They include, for example, different categories
\cite{las-book-cloud}:

\begin{itemize}
\item \textbf{REST Specification Frameworks} which  define
  REST service specifications for generating REST services in a
  language and framework independent manner such as Swagger 2.0
  \cite{openapi-2}, OpenAPI 3.0 \cite{openapi-3} and RAML
  \cite{raml-1}.
\item \textbf{REST programming language support} which include tools and services
  for targeting specific programming languages such as Flask Restful
  \cite{www-flask-restful}, and Django Rest \cite{www-django-rest} for Python.
\item \textbf{REST documentation-based tools} which are tools to document
  REST specifications. One such tool is Swagger \cite{www-swagger}.
\item \textbf{REST design support tools} which support the
  design process in developing REST services while defining reusable client and server that can be integrated and enhanced such as Swagger \cite{www-swagger} and other tools
  available at OpenAPI Tools \cite{www-openapi-tools} to generate code
  from OpenAPI specifications \cite{www-swagger-codegen}
\end{itemize}

Within our work reported here, we will heavily base our architecture
on REST. From this small discussion, it is evident that although the
concept of REST is easy to understand, a significant amount of
expertise is needed to apply it, which domain scientists may not be
interested in to know but keen on reusing without needing to know the
details.

\end{comment}

\subsection{OpenAPI}

One of the important aspects of generating REST services is a
language-independent formulation of REST services. For this reason,
the ``OpenAPI Specification (OAS) defines a standard,
language-agnostic interface to RESTful APIs which allows both humans
and computers to discover and understand the capabilities of the
service without access to source code, documentation, or through
network traffic inspection. When properly defined, a consumer can
understand and interact with the remote service with minimal
implementation logic \cite{openapi}.''

Hence this specification allows us to not only display the
documentation but also allows us to use it to generate the clients and
server stubs from it automatically. OpenAPI can be formulated as a
YAML Ain't Markup Language (YAML) \cite{www-yaml} file.

An OpenAPI definition can then be used by documentation generation
tools to display the API, code generation tools to generate servers
and clients in various programming languages, testing tools, and many
other use cases. One of the issues with using OpenAPI during our extensive prior work
is that it takes considerable effort to understand
the specification. Based on our experience of dessmination into
the community including agademia, industry, and government, it is a
formidable effort to learn and use it. This effort is often too big
to justify to be carried out by the AI domain experts.

\COMMENT{The
lessons from this effort that includes researchers, professionals,
graduate, and undergraduate students motivated this work.}

\subsection{Hybrid Multi-Cloud Computing with \Cloudmesh}\label{cloudmesh}

\BLOCKCENSOR{A large section was removed to fulfull double blind review
conditions.}{Cloud computing providers offer their customers on-demand self-service
computing resources that are rapidly elastic and accessible via broad
network access~\cite{nist-cloud-standard}.
They accomplish this through the economies of scale achieved by resource
pooling (serving multiple customers on the same hardware) and using
measured services for fine-grained customer billing \cite{nist-cloud-standard}.
Cloud providers offer these resources in multiple service models
including infrastructure as a service, platform as a service, software
as a service, and, recently, function as a service
\cite{nist-cloud-standard}.
These providers are rapidly offering new platforms and services ranging
from bare-metal machines to AI development platforms like Google's
TensorFlow Enterprise platform \cite{www-tensorflow-enterprise}, and AI services
such as Amazon's text-to-speech service \cite{amazon-polly}.

Customers can take advantage of cloud computing to reduce overhead
expenses, increase their speed and scale of service deployment, and
reduce development requirements by using cloud providers' platforms or
services. For example, customers' developing AI systems can utilize
clouds to handle big data inputs for which private infrastructure would
be too costly or slow to implement. However, having multiple competing
cloud providers leads to situations where service availability,
performance, and cost may vary. Customers must navigate these
heterogeneous solutions to meet their business needs while avoiding
vendor lock-in and managing organizational risk. This may require
comparing or using multiple cloud providers to meet various objectives.

Today's infrastructure deployments can benefit from a {\em hybrid
multi-cloud} strategy in which a mix of cloud-enabled services such as
computing, storage, and other services are integrated from on-premises
infrastructure, private cloud services, and a public cloud.


As pointed out earlier, \censor{Cloudmesh \cite{cloudmesh-manual}} is
a framework and toolkit that enables users to easily access hybrid
multi-cloud environments. \Cloudmesh is an evolution of previous tools
that have been used by many users. \Cloudmesh makes interacting with
clouds easy by creating a service mashup to access common cloud
services across numerous cloud platforms. \Cloudmesh contains a
sophisticated command shell, a database to store JSON objects
representing virtual machines, storage, and a registry of REST
services \censor{~\cite{cloudmesh-openapi}}.  \Cloudmesh has a
sophisticated plugin concept that is easy to use and leverages Python
namespaces while integrating plugins from different source code
directories \censor{~\cite{cloudmesh-github}}.  Installation
of \Cloudmesh is available for macOS, Linux, Windows, and RasberryOS
\censor{~\cite{cloudmesh-manual}}.

\Cloudmesh works with a variety of cloud providers, including Amazon Web
Services, Microsoft Azure, Google Cloud Platform, and Oracle's
OpenStack based providers such as the academic research serving
Chameleon Cloud \censor{~\cite{chameleon-cloud}}.

Recently we have also explored containers and microservices. The work
presented here summarizes some of this effort. With the help of a
plugin \censor{~{\em cloudmesh-openapi}} we can generate REST
services, including microservices and containers, to organize its
functions and code. In addition, cloudmesh can be distributed as a
container and used in a containerized environment. Through this
ability, cloudmesh services generated with \censor{{\em
cloudmesh-openapi}} can also be deployed on Kubernetes.
}

\section{Requirements}
\label{sec:requirements}

Next, we present the most critical requirements that motivated our
architecture and design. We start with a set of general requirements.

\begin{description}

\item[Leveraging new Python features.] Python is a very popular choice
with many data scientists. Our framework will leverage the newest
Python 3 features such as {\em Typing
Interface} \cite{www-python-typing} in order to increase robustness
and future-proofing of our code base.   

\item[Ability to be used within Jupyter Notebooks.] ~\\
The framework must be able to integrate with Jupyter notebooks as they
are very popular with today's data scientists. The functionality must
be easily accessible not only as part of Python programs but also
within Jupyter notebooks. This is of special importance also for cloud
services such as Google Colab \cite{google-colab} which for example,
offers cloud-based Notebooks.

\item[Easy of use] is a critical aspect of the framework that is to be
addressed from the start by allowing for ease of creation, ease of
deployment, and easy use of the generated services. This is
accompanied by easy to use command-line tools. 

\end{description}

Next, we list some more specific requirements that motivate our
architectural design.

\begin{description}

\item[Multi-Cloud Service Integration.] The framework must allow us to
integrate multiple cloud services, including IaaS, PaaS, and
SaaS. This also includes the ability to access AI-based services
offered by the various cloud providers. 

\item[Hybrid-Cloud Service Integration.] The framework must allow
integrating on-premise, private, and public clouds. 


\item[Generalized Analytics Service Generator.] We need a generalized
analytics service generator. The first step in the activity to
generate an analytics service is to provide an OpenAPI Service
generator. Our generator will allow us to define essential analytics
functions such as (a) uploading and downloading files to an analytics
service; (b) specifying the functionality through typing enhanced
Python functions; and (c) generating the code for the service.

\item[Generalized Analytics Service Deployment.]
After the service is generated, it needs to be deployed. For this step
we will be reusing the Cloudmesh deployment mechanism to instantiate
services on-demand on specified cloud providers such as AWS, Azure,
and Google.

\item[Generalized Analytics Service Invocation.]
The next step includes the invocation of the deployed services. While
analyzing some use cases, we identified that users often need to
invoke the same service many times to tune service parameters in a
quasi-realtime fashion while using parameters that can not be included
in the URL. Hence we will need to upload input parameters through
files if the simple typing data types provided by our proposed
framework is not sufficient.

\item[REST Services Architecture.]
As REST has become the most prominent architectural design principle,
our Generalized service architecture needs to be able to produce REST
services.

\item[Automated REST Service Generation for other Languages.] Our
framework must have provisions included that allow the integration
into other programming languages and, on the other hand, allows the
integration of services and functions developed in other languages. 
  
\item[Generalized Analytics Service Registry.]
As users and communities may develop many different services, we must
provide the ability to (a) find specifications of generalized
analytics services (b) find use-cases of generalized analytics
services (c) find infrastructure on which such services can be
deployed, and (d) find deployed analytics services. For this, we need
a registry that can be queried by the community.

\item[Generalized Composable Analytics Services.]
Services must be allowed to reuse other services to allow for easy
integration. Thus we need to make our services composable. This also
includes the choreography of the execution of such composable
services.


\end{description}

\section{Architecture}
\label{sec:architecture}

To satisfy our requirements, we have designed a neww multilayered
architecture delivering a framework and toolkit to allow the easy
generation and deployment of generalized AI-based REST services.

It contains two main layers. The first layer is concerned with
generating the REST services, while the second focuses on easy
deployment in a multi-cloud environment. However, as we also deal with
hybrid infrastructure, we allow in the second layer access to HPC and
local on premise resources. In addition, our architecture addresses
the creation of containers and their deployment in docker and
Kubernetes. This way our framework is not only capable of delopying
into cloud virtual machines, but also into other infrastructure
services, either offered locally or in the cloud. Both layers can be
accessed and controled via a convenient command shell and client. As
the resulting services are REST services, deployed services can easily
be accessed
from other resources via the automated generated REST API.
Next, we describe some of the
important features within each of the layers while starting with the
infrastructure deployment.

\subsection{IaaS Access Layer}

This layer allows us to deploy different infrastructure services on
demand while introducing an abstraction layer for compute resources
that allow IaaS access across the different platform offerings. Here,
we can leverage from the \Cloudmesh toolkit that provides us with the
basic interfaces to virtual machines to conveniently access in
homogeneous fashion AWS, Azure, Google, Oracle, and OpenStack. Access
to HPC and Bare metal can be integrated and has been showcased in the
past in cloudmesh. We also have prototyped in cloudmesh interfaces for
accessing compute resources via docker and Kubernetes.

All of the deployment can easily be managed through a simple client
shell that can also be used as a command line executor. This system is
one of the key components of \Cloudmesh and allows easy integration of
new commands and modules. This makes \Cloudmesh extensible, while
others can provide new functionality that can be accessed in the
command shell and command line interpreter. We use the \Cloudmesh
\censor{command shell} to integrate the functionality of the Generalized AI
Service (GAS) Generator and describe its functionality in more detail
next.

\subsection{GAS Generator}

The Generalized AI Service (GAS)
\COMMENT{\footnote{{\em GAS} the name GAS is
derived from two different common usages. First, it refers to
gasoline to fuel that we need to generate the services; the
second is an expandable material that fills the whole of a
container. If you have better ideas or analogies for naming our
framework, please get in contact with us. We love to hear from you!}}
Generator creates the REST service from a simple function or class
definition while utilizing the newest Python language features such as
documented typing information integrated in the program
specification. The GAS Generator provides us with the {\em fuel} that
is needed as part of the service deployment. This is manifested in a
number of artifacts for the deployment. The artifacts include a
specification derived from the Python program in OpenAPI format, the
server code that is derived from the OpenAPI format, and an optional
container specification file (e.g., Dockerfile). In addition, as we
expect that the service is going to be reused, we use a GAS Service
registry in which we record the specification description of the
service as well as deployment information on which the service ought
to be deployed. This deployment specification can be derived from
other prototype cloudmesh components such as cloudmesh-frugal, which
can obtain resources based on minimal cost. We have not explicitly
included this component in our architecture picture as we have not
used it as part of our benchmarks that we describe later.

% \TODO{mod image to add a backlink to the user}

\OneFIGURE
    {openapi-arch-new-2.pdf}
    {Layered architecture of the cloudmesh OpenAPI framework.}
    {fig:arch-new}
    {1.0}

\subsection{GAS Workflow}

To showcase why the framework is so useful for data scientists and reduces
the steps needed to generate AI services, we are
contrasting the definition workflow that a scientist undergoes while
using OpenAPI without and with GAS Generator
in \Figure{fig:flow-schema} and \Figure{fig:flow-function}.

The workflow in \Figure{fig:flow-schema} showcases a typical workflow
as promoted by the developers of Swagger
codegen \cite{www-swagger-codegen}. The user identifies from his use
case an OpenAPI schema that is used to generate the code. However,
this is an unnecessary high entry barrier as the creation of these
schemas is complex. While using the swagger code generator, a variety
of code stubs in many different languages can be created. The code
generated requires an unnecessary high entry barrier as we next need
to identify how and where we include an implementation of a function
that we want to expose as a REST service. Once complete, the rest of
the activity requires the remaining steps to be executed by hand.

Next, we like to contrast this with our much-simplified approach. As
we know, the data scientists have the knowledge to write a Python
function (or class); we simply leverage this expertise and take the
function (or classs) and provide it as input (fuel) to the GAS
generator. This is done with a simple one-line command invocation that
includes the filename of the Python program in which the function (or
class) is defined. The scientist does not have to learn REST, the
scientist does not have to look into a code stub that is generated for,
the scientist does not even have to know how to instantiate or run
the service. Furthermore, the scientist does not have to know about
any security as we have added features to the code to leverage the
existing security mechanism as a simple flag to the GAS Generator
command line instantiation. This simplification allows the scientist
to develop REST services in minutes rather than a month as the entry
barrier is very low. Additionally, as we are integrated with \Cloudmesh
to facilitate
infrastructure deployments, the instantiation of the services can also
be done with a one line command under the assumption the scientist has
accounts registered with the appropriate IaaS or PaaS provider. The scientist
clearly can utilize this multi-cloud experinece which would otherwise
to complex to achieve due to the high entry barrier. Furthermore we see that
the overall workflow complexity is reduced as the scientist can just define the
function up front instead of switching between code generation and code
integration.

\begin{figure}[htb]

\begin{subfigure}{.48\columnwidth}
    \centering
    \begin{minipage}[b]{.6\columnwidth}
    \begin{center}
\begin{adjustbox}{width=1.0\columnwidth}
\begin{footnotesize}
\forestset{
  skan tree/.style={
    for tree={
      drop shadow,
      text width=3cm,
      %grow'=0,
      %growth=south,
      rounded corners,
      draw,
      top color=white,
      bottom color=blue!20,
      edge={Latex-},
      child anchor=parent,
      %parent anchor=children,
      anchor=parent,
      tier/.wrap pgfmath arg={tier ##1}{level()},
      s sep+=2.5pt,
      l sep+=2.5pt,
      edge path'={
        (.child anchor) -- (!u.parent anchor)
      },
      node options={ align=center },
    },
    before typesetting nodes={
      for tree={
        content/.wrap value={\strut ##1},
      },
    },
  },
}
\begin{forest}
  skan tree
  [Usecase, \nwhite
    [Schema (complex), \nwhite
       [Code Generator, \ngreen
          [Implementation (complex), \nwhite
          [Function (complex), \nwhite
          [Service, \ngreen
             [Deployment (not provided), \nred
               [Verification (not provided), \nred
                  [Hosting (not provided), \nred
                  ]
               ]
             ]
            ]
            ]
         ]
      ]
    ]
  ]
\end{forest}
\end{footnotesize}
\end{adjustbox}
\end{center}
%\vspace{-0.8cm}

    \end{minipage}
    \caption{Schema-based component flow to specify an analytics service.}
    \label{fig:flow-schema}
\end{subfigure}
\begin{subfigure}{.48\columnwidth}
  \centering
  \vspace{1.2cm}
  \begin{minipage}[b]{.6\columnwidth}
  \input{flow-function}
    \end{minipage}  
  \caption{New \Cloudmesh function-based component flow to specify an analytics service.}
  \label{fig:flow-function}
\end{subfigure}

\caption{Contrasting the new apprach to autogenerate the REST service from functions. The white boxes represent user input, the green boxes represent tools that automatically generate, and the red boxes represent service deployment.}

\end{figure}


\subsection{Scripting as Fuel for the GAS Generator}

Next, we demonstrate through two examples the simplicity of this
aproach in praxis. In our example, we define a function returning a
result. We chose a simple add function and list the code related to it
in \Figure{fig:function}. Alternatively we could define it as a class
showcased in \Figure{fig:class}. Other
than using the typing annotations, the example is
just a regular Python program. It can be tested locally on the system
to check its functionality before we generate the service.

\Figure{fig:deploy-function} shows how to generate and deploy the
service. As this process is the same for the class-based definition
and only differs in the filename, we omitted it to include an explicit
listing of the access method for it. 

Once the service is deployed the {\em curl} calls
in \Figure{fig:function-curl} and \Figure{fig:class-curl} showcase how
to interact with the service from the command line. Naturally, one can
use any programming language that has built-in libraries for HTTP
requests to interface with the service (such as {\em requests} in
Python \cite{www-python-requests}).  Once we execute the following
lines in a terminal, the result of the the addition will be calculated
in the REST service, and it is returned.

\newcommand{\FONT}{\tiny}


\begin{figure}[htb]
\begin{lstlisting}[language=Python,
                   basicstyle=\ttfamily\FONT,
                   numbers=left,                   
                   numbersep=5pt,
                   xleftmargin=5mm]
def add(x: float, y: float) -> float:
    """
    adding x and y.
    :param x: x value
    :param y: y value
    :return: result
    """
    result = x + y
    return result
\end{lstlisting}
\caption{Defining an analytics function that is used to generate a REST service.}
\label{fig:function}
\bigskip

\begin{lstlisting}[language=Python,
                   basicstyle=\ttfamily\FONT,
                   numbers=left,                   
                   numbersep=5pt,
                   xleftmargin=5mm]
class Calculator:

    @classmethod
    def multiply(cls, x: int, y: int) -> int:
        """
        Multiply int by int and return an int.

        :param x: the value of input #1
        :param y: the value of input #2
        :return: result of multiplying x by y
        """
        return x * y

    @classmethod
    def divide(cls, x: int, y: float) -> float:
        """
        Divide int by float and return a float.

        :param x: the value of input #1
        :param y: the value of input #2
        :return: result of dividing x by y
        """
        return x / y

if __name__ == '__main__':
    calc = Calculator()
    print("multiply 1 * 2: ",  calc.multiply(1, 2))
    print("divide 6 / 3.14: ",  calc.divide(6, 2.3))

\end{lstlisting}

\caption{Defining an analytics function with the help of class methods to generate a REST service with multiple functions.}
\label{fig:class}


\begin{lstlisting}[language=bash,
                   firstnumber=10,
                   basicstyle=\ttfamily\tiny,
                   numbers=left,                   
                   numbersep=5pt,
                   xleftmargin=5mm]
$ cms openapi generate add --filename=./add.py
$ cms openapi server start ./add.yaml 
$ curl \ 
 -X GET "http://localhost:8080/add?x=1&y=2" -H "accept: text/plain"
# This command returns
> 3.0
\end{lstlisting}
\caption{Generating, deploying, and invoking the REST service. {\color{white}aaaaaaaa}}
\label{fig:deploy-function}
\label{fig:function-curl}



\bigskip

\begin{lstlisting}[language=Python,
                   firstnumber=29,
                   basicstyle=\ttfamily\FONT,
                   numbers=left,                   
                   numbersep=5pt,
                   xleftmargin=5mm]
$ curl -X GET "http://localhost:8080/multiply?x=1&y=2" -H "accept: text/plain"
$ curl -X GET "http://localhost:8080/divide?x=6&y=3.14" -H "accept: text/plain"
\end{lstlisting}

\caption{Defining an analytics function with the help of class methods to generate a REST service with multiple functions.}
\label{fig:class-curl}

\end{figure}


\subsection{GAS Security}

As we leverage OpenAPI and automatically generated OpenAPI services,
it is possible to leverage security mechanisms from the underlying
service implementation. To showcase this ability, we added basic
authentication into our framework as an example
configuration. However, it is certainly possible to extend this as the
services we use also support OAuth, ApiKey Authentication, Bearer
Authentication (JWT), and HTTPS
support \cite{connexion-security}\censor{\cite{cloudmesh-openapi}}.

To demonstrate basic authentication, a cloudmesh user can create an
OpenAPI server whose endpoints are only accessible as an authorized
user. Currently, when basic auth is used, all endpoints are secured
with this method. In future versions, we intend to allow securing
selected methods.  An example of the usage of basic auth is provided
on our Web page at \censor{\cite{www-cloudmesh-openapi-security}}.

\section{Deployment Scenarios}

Due to the versatility of REST and our ability to integrate with a
variety of infrastructure services, a rich set of deployment scenarios
is possible. Two important scenarios related to single and multiple
service provider deployments are discussed next.

\subsection{Single Cloud Provider Hosted AI Service}

In this scenario, a user deploys GAS services on a virtual
machine from a cloud provider and uses it to host auto-generated,
RESTful, AI services. Next, the scientist constructs an AI service as
a set of Python functions that implement a workflow, for example,
downloading data from a remote server, training an AI model, uploading
a new sample for prediction, and running a prediction on that
sample. After the deployment, the service is accessible using standard
HTTP request methods. In \Figure{fig:1} we show a remote client that
accesses such a typical AI service workflow.  Here the service is just
deployed on a virtual machine from a single cloud provider. GAS
provides the choice on which infrastructure provider to place the
service. Through our security mechanism, the service can either be
exposed to the public or to authenticated and authorized users.

\OneFIGURE
  {gas-1.pdf}
  {Example AI Service Workflow to obtain data, train, upload data for prediction, as well as the interaction with it.}
  {fig:1}
  {0.5}
    
\subsection{Multi-Cloud Hosted AI Service}

In our next scenario, we like to depict that it is possible to deploy
the same service on multiple clouds through the use of our
sophisticated but easy to use command clients. Detailed information
about the exact commands are provided in our
manual \censor{\cite{cloudmesh-manual}}. Through this we can, for
example, evaluate suitable providers for our deployment through
benchmarks as we show next. Thus, the scientist not only obtains the
ability with GAS Generator to develop and deploy services, but also to
evaluate their performance on a variety of infrastructures. An example
is provided in \Figure{fig:2} where a scientist deploys
a service on AWS, Azure, and Google. As they are asynchronous
services, the scientist can query the services simultaneously and
gathers responses and benchmarks. Obviously, this can be used in a
real scenario to integrate compute resources from multiple providers
that can be accessed via our GAS services. It also allows specific
adaptations such as the integration of cloud AI services with one
provider that are not accessible in another. Hence the framework can
also be utilized to benchmark unique services that are offered by a
particular provider, or if they are offered by more than one, they can
be comparatively evaluated.


\OneFIGURE
  {gas-2.pdf}
  {Mult-Cloud AI Services: A client simultaneously accesses an AI service hosted
   on three separate cloud providers, AWS, Azure, and Google, to benchmark
   provider performance.}
  {fig:2}
  {0.5}

\section{Benchmark}
\label{sec:benchmark}

In this section we describe our benchmark results.

\subsection{Application}

Our benchmarks are based on pytests that integrate Scikit-learn
artificial intelligent algorithms to allow replication. These pytests
are run on different cloud services to benchmark end evaluate how they
perform. The Benchmarks include services to conduct data transfers,
model train, model prediction, and more. A report is automatically generated. 

Within Scikit-learn we have chosen the {\bf\em Eigenfaces SVM Facial
Recognition} example as it represents a very common data science usage
pattern. This example conducts a facial recognition that first
utilizes principle component analysis (PCA) to generate eigenfaces
from the training image data, and then trains and tests an SVM
model \cite{www-skikit-learn-faces}. It uses the real world
{\em Labeled Faces in the Wild} dataset consisting of labeled images
of famous individuals gathered from the internet \cite{faces-data}.

This project uses deafault datasest from Scikit-learn to allow others
to easily test and replicate the GAS approach. While using these
examples, it will be possible to easily adpt it to use other data
sources, as well as algorithms.


\subsection{Infrastructure}

We want to compare service
deployments on virtual machines that are hosted on various cloud
providers. We have chosen to select similar virtual machines for
conducting the benchmark. This includes AWS \cite{www-aws},
Azure\cite{www-azure}, and Google \cite{www-google}.

In addition, we are performing some bare metal experiments on two
Raspberry PI clusters, one with Raspberry PI4's and the other with
Raspberry PI 3b+'s. The latter has a management node, a PI 4, and
worker nodes that are PI 3b+. The inclusion of the Raspberry platform
was important to us as it demonstrates the capability of IoT and Edge
computing devices that may become more prevalent in the future for
delegating tasks to the edge. We further provide a docker container
for a comparison of containerized services.

For this work we provide two application benchmarks for the Eigenfaces
SVM. We have chosen this example to demonstrate the feasability of the
GAS framework.  The first deploys and measures the AI service on a
single cloud provider at a time
(see \ref{single-cloud-provider-service-benchmarking}), and the second
deploys a multi-cloud (see \ref{sec-multi-benchmark}) AI service
measuring the service across the clouds in parallel.


%  \textbf{Pipelined ANOVA SVM}: An example code that shows a pipeline
%  successively running a univariate feature selection with anova and
%  then a SVM of the selected features \cite{www-skikit-learn-pipeline}


\subsection{VM Selection}\label{vm-selection}

When benchmarking cloud performance, it is important to identify and
control VM deployment parameters. This allows one to analyze
comparable service offerings, or identify opportunities for
performance improvement by varying deployment features such as machine
size, location, network, or storage hardware. These benchmark examples
aimed to create similar machines across all three clouds, and measure
their service performance. See \Table{tab:iaas} for a summary of the
parameters controlled in these benchmark examples.

One key component is the virtual machine size, which determines the
number of vCPUs, the amount of memory, attached storage types, and
resource sharing policies. Resource sharing policies include shared
core machine varieties—which providers offer at less expensive
rates—that allow the virtual machines to burst over its base clock
rate in exchange for credits or the machine's inherent bursting factor
\cite{amazon-instances,google-instances}. For this example, we chose
three similar machine sizes that had comparable: vCPUs, underlying
processors, memory, price, and were not a shared core variety. We
installed the same Ubuntu 20.04 operating system on all three clouds.

Another factor that can affect performance, particularly in network
latency, is the zone and region selected. We deploy all benchmark
machines to zones on the east coast of the United States. This helps
control variations caused by network routing latency and provides more
insight into the inherent network performance of the individual cloud
services.

%\rowcolors{2}{gray!25}{white}

Because cloud providers can observe varying loads during the day, the
benchmark execution time is another parameter to control. In our
single cloud provider benchmark for the Eigenfaces SVM example, clouds
were tested at least twice and were run sequentially between the hours
of approximately 19:45 pm EST and 03:30 am EST. \COMMENT{starting with Google and
ending with Azure.} \COMMENT{In the Eigenfaces SVM example, only 60 runs were
conducted on Azure due to a failed VM deployment caused by factors
outside of the benchmark script's control.} Compared to our single
cloud provider benchmark, our multi-cloud benchmark benefits from all
clouds being tested at the same time.

\begin{table}
  
\caption{Controlled VM parameters for cloud benchmarks.}
\label{tab:iaas}

\resizebox{1.0\columnwidth}{!}{
\begin{tabular}[]{@{}llll@{}}
\toprule
 & AWS & Azure & Google \tabularnewline
\midrule
%\endhead
Size (flavor) & m4.large & Standard\_D2s\_v3 & n1-standard-2 \tabularnewline
vCPU & 2 & 2 & 2 \tabularnewline
Memory (GB) & 8 & 8 & 7.5 \tabularnewline
Image & ami-0dba2cb6798deb6d8
& \begin{minipage}[t]{0.40\columnwidth}\raggedright
Canonical:0001-com-ubuntu-server-focal:20\_04-lts:20.04.202006100\strut
\end{minipage} & 
ubuntu-2004-lts
\tabularnewline
OS & Ubuntu 20.04 LTS & Ubuntu 20.04 LTS & Ubuntu 20.04 LTS \tabularnewline
Region & us-east-1 & eastus & us-east1 \tabularnewline
Zone &  N/A &  N/A & us-east1-b \tabularnewline
Price (\$/hr) &  0.1 & 0.096 & 0.0949995 \tabularnewline
%Runs/Test & 90 & 60 & 90\tabularnewline
\bottomrule
\end{tabular}
}
\bigskip

\caption{Raspberry Pi and Docker Specifications}
\label{tab:pi}
\resizebox{1.0\columnwidth}{!}{
\begin{tabular}[]{lllll}
\toprule 
&        &      &  Docker     & MacBook \tabularnewline
& Pi 3B+ & Pi 4 & (On MBP) & Pro i5 3.1GHz\tabularnewline
\midrule
%\endhead
Cores & 4 & 4 & 2&2\tabularnewline
Memory (GB) & 1 & 8  & 2&8\tabularnewline
OS & Raspberry OS 10 & Raspberry OS 10 & Ubuntu 20.04 LTS & macOS \tabularnewline
Version & Kernel 5.4.51  & Kernel 5.4.51 & NA& Big Sur \tabularnewline
Purchase Cost (\$) & 51.99 & 109.99 & NA& NA\tabularnewline
Energy Cost (\$/year) & 5.36 & 6.73 & NA & NA\tabularnewline
Price (\$/hr) &  0.0065 & 0.0133 & NA & NA\tabularnewline
%Runs/Test & 30 &  30 & 1 &\tabularnewline
\bottomrule
\end{tabular}
}


\smallskip
{\footnotesize The Price is the purchase cost and 1yr energy cost,
amortized over a year and given for each hour of the year.}
\end{table}

\subsection{Single Cloud Provider AI Service Benchmark.}
\label{single-cloud-provider-service-benchmarking}

The benchmark script for the Eigenfaces SVM example uses \Cloudmesh to
create virtual machines and set up the GAS OpenAPI services
sequentially across the three measured clouds, Amazon, Azure, and
Google. After the script sets up the environment, it runs a series of
pytests that generate and launch the Eigenfaces-SVM OpenAPI service,
and then conducts runtime measurements of various service
functions. Also, we run the same pytests on two Raspberry Pi models, a
MacBook Pro running a Docker container, and a bare metal MacBook Pro
to demonstrate \Cloudmesh OpenAPI's flexibility for multi-platform
use.

The benchmark runs the pytest in two configurations. After the benchmark
the script sets up a virtual machine environment, it runs the first pytest
locally on the OpenAPI server and measures five runtimes:

\begin{enumerate}
\def\labelenumi{\arabic{enumi}.}
\item \textbf{download data:} Download and extraction of remote image data from
  ndownloader.figshare.com/files/5976015
\item \textbf{train:} 
  The model training time when run as an OpenAPI service
\item \textbf{scikitlearn train:} 
  The model training time when run as the Scikit-learn example without
  OpenAPI involvement
\item \textbf{upload local:} 
  The time to upload an image from the server to itself
\item \textbf{predict local:} 
  The time to predict and return the target label of the uploaded image
\end{enumerate}

The benchmark runs the second pytest iteration as a remote client and
interacts with the deployed OpenAPI service over the internet. It
tests two runtimes:

\begin{enumerate}
\def\labelenumi{\arabic{enumi}.}
\item \textbf{upload remote:} 
  The time to upload an image to the remote OpenAPI server
\item \textbf{predict remote:} 
  The time to run the predict function on the remote OpenAPI server, and
  return the target label of the uploaded image
\end{enumerate}

In \Figure{fig:download} we compare the download and extraction time
of the labeled faces in the wild data set. This data set is
approximately 233 MBs compressed, which allows us to measure a
relatively small but non-trivial data transfer.  Lower transfer times imply the cloud has
higher throughput from the data server, less latency to the data
server, or that the cloud has a better performing internal
network. The standard deviation is displayed to compare the variation
in the download times. Because the difference between commercial and
residential internet speeds dominates the function runtime, we do not
compare the clouds to the Pi models, MacBook, or docker container.
\begin{comment}
In  \Figure{fig:download} we show the same plot without the Pi and
docker results to allow a closer comparison of the three comparable clouds.
\end{comment}

\begin{comment}
\TwoFIGURES
    {sample-graph-1-pi1.pdf}
    {Runtime for downloading the data used in the Eigenfaces SVM benchmark.}
    {fig:download-pi}
    {sample-graph-1.pdf}
    {Closeup of the Runtime for downloading the data used in the Eigenfaces SVM benchmark while excluding the runtime for the Raspberry PI.}
    {fig:download}
\end{comment}

\OneFIGURE
    {sample-graph-1.pdf}
    {Runtime for downloading the data used in the Eigenfaces SVM benchmark.}
    {fig:download}
    {1.0}
    
In \Figure{fig:train-pi} we depict the training time of the
Eigenfaces-SVM model both as an OpenAPI service and as the basic
Scikit-learn example. This allows us to measure the runtime overhead
added by OpenAPI compared to the source example. Here, the two
functions are identical except that the OpenAPI train function makes
an additional function call to store the model to disk. This is
necessary to share the model across the train and predict
functions. In the figure there are two bars per cloud provider. The
blue bars are the training time of the model when hosted as
a \Cloudmesh OpenAPI function. The orange bars are the training time
of the Scikit-learn example code without \Cloudmesh OpenAPI
involvement. The bars plot the mean runtimes and the error bar
reflects the standard deviation of the runtimes.

\COMMENT{In \Figure{fig:train}
we show the same plot without the Pi models, MacBook, and docker
results to allow a closer comparison of the three comparable clouds.}

%\TwoFIGURES

\OneFIGURES
    {sample-graph-2-pi1.pdf}
    {XXXX Runtime for training on the data used in the Eigenfaces SVM benchmark.}
    {fig:train-pi}
    {1.0}
    
    %{sample-graph-2.pdf}
    %{Closeup of the Runtime for training on  the data used in the Eigenfaces SVM benchmark without the data for the Pi.}
    %{fig:train}


In \Figure{fig:upload-pi} we present the time to upload an image to
the server both from itself and from a remote client. This allows us
to compare the function runtime as experienced by the server, and as
experienced by a remote client. The difference helps determine the
network latency between the benchmark client and the cloud service. In
the figure, there are two bars per cloud provider. The blue bars are
the runtime of the upload function as experienced by the server, and
the orange as experienced by the remote client. The bars plot the mean
runtimes and the error bar reflects the standard deviation of the
runtimes. For the Pi models, MacBook, and docker container, we only
measure the local function runtime.

%\TwoFIGURES
\OneFIGURE
    {sample-graph-3-pi1.pdf}
    {Runtime for uploading the data used in the Eigenfaces SVM benchmark.}
    {fig:upload-pi}
    {1.0}
    
    %{sample-graph-3.pdf}
    %{Closeup of the Runtime for uploading the data used in the Eigenfaces SVM benchmark without the data for the Pi.}
    %{fig:upload}

In \Figure{fig:predict-pi} we measure the time to call the predict
function on the uploaded image. Again we run this once from the local
server itself, and a second time from a remote client to determine
client and server runtimes. In the figure, there are two bars per
cloud provider. The blue bars are the run time of the predict function
as experienced by the server, and the orange as experienced by the
remote client to measuer its overhead. The bars plot mean runtimes and
the error bar reflects
the standard deviation of the runtimes. For the Pi models, MacBook,
and docker container, we only measured the local function runtime.

%\TwoFIGURES
\OneFIGURE
    {sample-graph-4-pi1.pdf}
    {Runtime for the prediction used in the Eigenfaces SVM benchmark.}
    {fig:predict-pi}
    {1.0}

    %{sample-graph-4.pdf}
    %{Closeup of the Runtime for the prediction used in the Eigenfaces SVM benchmark without the data for the Pi.}
    %{fig:predict}

\Table{tab:2} presents a full listing of test results. For the upload
    and predict tests, the 'type' column denotes whether the test was
    run locally (server runtime) or remote (client runtime). 

In \Table{tab:cost} we present a cost analysis of the service
functions. The analysis uses the price from \Table{tab:iaas}
and \Table{tab:pi}. The price for the cloud virtual machines are based
on provider advertised costs, while the price for the Pi models are
based on the hardware cost and one year of energy cost amortized for
one year. This does not include other costs such as cooling,
networking, or real estate. For the Pi energy cost we assume a full
and constant load. We utilize power consumption benchmarks
from \cite{pi-power} and \censor{Indiana} residential kWH cost
from \cite{indiana-energy} to calculate the expected Energy Cost per
year. We calculate the cost to run each function and compare the
clouds and Raspberry Pi 4 to the Raspberry Pi 3b+. We compare the
percent runtime decrease from the Pi 3b+ to the clouds and Raspberry
Pi4, and the percent cost increase from the Pi 3b+ to the clouds and
Raspberry Pi 4.

\begin{comment}

\begin{table}[h]
%\caption{example table with vertical labels}
example for vertical labels
\centering
\begin{tabular}{clllrrrr}
\toprule
type & \multicolumn{1}{c}{test} & \multicolumn{1}{c}{cloud} & \multicolumn{1}{c}{mean} & \multicolumn{1}{c}{min} & \multicolumn{1}{c}{max} & \multicolumn{1}{c}{std}\\ 
\hline
\multirow{3}{*}{\rotatebox[origin=c]{90}{\parbox[c]{1cm}{\centering local}}} & 
\multirow{3}{*}{\rotatebox[origin=c]{90}{\parbox[c]{1cm}{\centering down\-load}}} 
  & aws &2&3&4&5\\
& & azure & 2&3&4&5\\
& & google &3&3&4&5\\ 
\hline
\multirow{3}{*}{\rotatebox[origin=c]{90}{\parbox[c]{1cm}{\centering local}}} & 
\multirow{3}{*}{\rotatebox[origin=c]{90}{\parbox[c]{1cm}{\centering predict}}} 
  & aws &2&3&4&5\\
& & azure & 2&3&4&5\\
& & google &3&3&4&5\\ 
\hline
\multirow{3}{*}{\rotatebox[origin=c]{90}{\parbox[c]{1cm}{\centering local}}} & 
\multirow{3}{*}{\rotatebox[origin=c]{90}{\parbox[c]{1cm}{\centering remote}}} 
  & aws &2&3&4&5\\
& & azure & 2&3&4&5\\
& & google &3&3&4&5\\ 
\hline
\end{tabular}
\end{table}
\end{comment}


\begin{table}[htb]
  
\caption{Test results for the Eigenfaces SVM single cloud provider benchmark.}
\label{tab:2}

\resizebox{0.9\columnwidth}{!}{
\begin{tabular}{lllrrrr}
\toprule
              test &    type &     cloud &    mean &     min &     max &   std \\
\midrule
     download data &   local &       aws &   20.58 &   17.23 &   31.80 &  2.77 \\
     download data &   local &     azure &   20.81 &   13.56 &   42.70 &  6.94 \\
     download data &   local &    docker &  820.98 &  820.98 &  820.98 &  0.00 \\
     download data &   local &    google &   18.00 &   17.06 &   19.38 &  0.48 \\
     download data &   local &    pi 3b+ &  130.17 &  123.84 &  149.40 &  5.39 \\
     download data &   local &      pi 4 &   47.67 &   43.43 &   75.60 &  5.72 \\
\midrule
           predict &   local &       aws &    0.03 &    0.02 &    0.05 &  0.00 \\
           predict &   local &     azure &    0.02 &    0.01 &    0.03 &  0.00 \\
           predict &   local &    docker &    0.03 &    0.03 &    0.03 &  0.00 \\
           predict &   local &    google &    0.03 &    0.01 &    0.06 &  0.00 \\
           predict &   local &  mac book &    0.12 &    0.12 &    0.12 &  0.00 \\
           predict &   local &    pi 3b+ &    0.12 &    0.10 &    0.14 &  0.01 \\
           predict &   local &      pi 4 &    0.08 &    0.08 &    0.08 &  0.00 \\
\midrule
           predict &  remote &       aws &    0.40 &    0.26 &    0.80 &  0.18 \\
           predict &  remote &     azure &    0.36 &    0.24 &    0.60 &  0.13 \\
           predict &  remote &    google &    0.36 &    0.27 &    0.82 &  0.16 \\
\midrule
 scikitlearn train &   local &       aws &   35.89 &   35.11 &   46.45 &  1.77 \\
 scikitlearn train &   local &     azure &   40.13 &   34.95 &   43.96 &  3.29 \\
 scikitlearn train &   local &    docker &   53.76 &   53.76 &   53.76 &  0.00 \\
 scikitlearn train &   local &    google &   42.13 &   41.77 &   42.49 &  0.13 \\
 scikitlearn train &   local &  mac book &   32.53 &   32.53 &   32.53 &  0.00 \\
 scikitlearn train &   local &    pi 3b+ &  222.63 &  209.18 &  231.90 &  7.87 \\
 scikitlearn train &   local &      pi 4 &   88.32 &   87.78 &   89.14 &  0.33 \\
\midrule
             train &   local &       aws &   35.72 &   34.91 &   46.50 &  1.73 \\
             train &   local &     azure &   40.28 &   35.30 &   47.50 &  3.32 \\
             train &   local &    docker &   54.72 &   54.72 &   54.72 &  0.00 \\
             train &   local &    google &   42.04 &   41.52 &   45.93 &  0.71 \\
             train &   local &  mac book &   33.82 &   33.82 &   33.82 &  0.00 \\
             train &   local &    pi 3b+ &  222.61 &  208.56 &  233.48 &  8.40 \\
             train &   local &      pi 4 &   88.59 &   87.83 &   89.35 &  0.32 \\
\midrule
            upload &   local &       aws &    0.01 &    0.01 &    0.01 &  0.00 \\
            upload &   local &     azure &    0.01 &    0.00 &    0.01 &  0.00 \\
            upload &   local &    docker &    0.02 &    0.02 &    0.02 &  0.00 \\
            upload &   local &    google &    0.01 &    0.01 &    0.01 &  0.00 \\
            upload &   local &  mac book &    0.02 &    0.02 &    0.02 &  0.00 \\
            upload &   local &    pi 3b+ &    0.09 &    0.04 &    0.48 &  0.08 \\
            upload &   local &      pi 4 &    0.02 &    0.02 &    0.02 &  0.00 \\
\midrule
            upload &  remote &       aws &    0.43 &    0.16 &    1.13 &  0.21 \\
            upload &  remote &     azure &    0.32 &    0.15 &    0.50 &  0.15 \\
            upload &  remote &    google &    0.31 &    0.18 &    0.73 &  0.18 \\
\bottomrule
\end{tabular}
}
\bigskip

\caption{Test results for the Eigenfaces SVM benchmark deployed
 as a multi-cloud service.}
\label{tab:3}

\resizebox{0.9\columnwidth}{!}{
\begin{tabular}{lllrrrr}
\toprule
test & type & cloud & mean & min & max & std \\
\midrule
download data & remote & aws    & 20.51 & 17.57 & 34.42 & 3.82 \\
download data & remote & azure  & 18.60 & 13.49 & 32.65 & 4.53 \\
download data & remote & google & 17.90 & 17.13 & 21.86 & 0.85 \\
\midrule
predict       & remote & aws	&  4.15 &  3.59 &  5.42 & 0.57 \\
predict       & remote & azure 	&  3.93 &  3.40 &  6.65 & 0.74 \\
predict       & remote & google &  4.13 &  3.74 &  6.37 & 0.60 \\
\midrule
train 	      & remote & aws 	& 35.61 & 35.24 & 39.53 & 0.73 \\
train 	      & remote & azure 	& 35.89 & 35.08 & 40.00 & 0.95 \\
train 	      & remote & google & 41.98 & 41.58 & 45.71 & 0.71 \\
\midrule
upload 	      & remote & aws 	& 10.08 &  4.89 & 16.52 & 4.38 \\
upload 	      & remote & azure 	&  8.46 &  4.72 & 13.92 & 4.05 \\
upload 	      & remote & google &  8.87 &  5.39 & 15.44 & 4.52 \\
\bottomrule
\end{tabular}
}


\end{table}

\begin{table}[htb]
\caption{Cost Analysis of function runtimes with \% cost increase and \% runtime decrease relative to the Raspberry Pi 3B+. }
\label{tab:cost}
\resizebox{1.0\columnwidth}{!}{
\begin{tabular}{lllrrrr}
\toprule
              test &    type &     cloud &    mean &     cost &  \% runtime decrease &  \% cost increase \\
\midrule
     download data &   local &       aws &   20.58 & 5.72e-04 &               NA &           NA \\
     download data &   local &     azure &   20.81 & 5.55e-04 &               NA &           NA \\
     download data &   local &    google &   18.00 & 4.75e-04 &               NA &           NA \\
\midrule
           predict &   local &       aws &    0.03 & 8.33e-07 &               75.00 &           281.87 \\
           predict &   local &     azure &    0.02 & 5.33e-07 &               83.33 &           144.39 \\
           predict &   local &    google &    0.03 & 7.92e-07 &               75.00 &           262.77 \\
           predict &   local &  mac book &    0.12 &      NA &                0.00 &              NA \\
           predict &   local &    docker &    0.03 &      NA &               75.00 &              NA \\
           predict &   local &      pi 4 &    0.08 & 2.96e-07 &               33.33 &            35.68 \\
           predict &   local &    pi 3b+ &    0.12 & 2.18e-07 &                0.00 &             0.00 \\
\midrule
           predict &  remote &       aws &    0.40 & 1.11e-05 &                 NA &              NA \\
           predict &  remote &     azure &    0.36 & 9.60e-06 &                 NA &              NA \\
           predict &  remote &    google &    0.36 & 9.50e-06 &                 NA &              NA \\
\midrule
 scikitlearn train &   local &       aws &   35.89 & 9.97e-04 &               83.88 &           146.24 \\
 scikitlearn train &   local &     azure &   40.13 & 1.07e-03 &               81.97 &           164.32 \\
 scikitlearn train &   local &    google &   42.13 & 1.11e-03 &               81.08 &           174.60 \\
 scikitlearn train &   local &  mac book &   32.53 &      NA &               85.39 &              NA \\
 scikitlearn train &   local &    docker &   53.76 &      NA &               75.85 &              NA \\
 scikitlearn train &   local &      pi 4 &   88.32 & 3.27e-04 &               60.33 &           -19.26 \\
 scikitlearn train &   local &    pi 3b+ &  222.63 & 4.05e-04 &                0.00 &             0.00 \\
\midrule
             train &   local &       aws &   35.72 & 9.92e-04 &               83.95 &           145.10 \\
             train &   local &     azure &   40.28 & 1.07e-03 &               81.91 &           165.33 \\
             train &   local &    google &   42.04 & 1.11e-03 &               81.11 &           174.04 \\
             train &   local &  mac book &   33.82 &      NA &               84.81 &              NA \\
             train &   local &    docker &   54.72 &      NA &               75.42 &              NA \\
             train &   local &      pi 4 &   88.59 & 3.28e-04 &               60.20 &           -19.01 \\
             train &   local &    pi 3b+ &  222.61 & 4.05e-04 &                0.00 &             0.00 \\
\midrule
            upload &   local &       aws &    0.01 & 2.78e-07 &               88.89 &            69.72 \\
            upload &   local &     azure &    0.01 & 2.67e-07 &               88.89 &            62.93 \\
            upload &   local &    google &    0.01 & 2.64e-07 &               88.89 &            61.23 \\
            upload &   local &  mac book &    0.02 &      NA &               77.78 &              NA \\
            upload &   local &    docker &    0.02 &      NA &               77.78 &              NA \\
            upload &   local &      pi 4 &    0.02 & 7.40e-08 &               77.78 &           -54.77 \\
            upload &   local &    pi 3b+ &    0.09 & 1.64e-07 &                0.00 &             0.00 \\
\midrule
            upload &  remote &       aws &    0.43 & 1.19e-05 &                 NA &              NA \\
            upload &  remote &     azure &    0.32 & 8.53e-06 &                 NA &              NA \\
            upload &  remote &    google &    0.31 & 8.18e-06 &                 NA &              NA \\
\bottomrule
\end{tabular}
}
\end{table}

\subsection{Multi-Cloud AI Service Benchmark}
\label{sec-multi-benchmark}

In this benchmark, our script first acquires VMs, installs \Cloudmesh
OpenAPI, and launches the Eigenfaces SVM AI service on three separate
cloud providers. \COMMENT{Because \Cloudmesh has limited parallel computing
support, the script deploys the VMs in a serial manner.} After the
services are running, we then run our tests in a parallel manner as
depicted in \Figure{fig:2}. \COMMENT{Testing in parallel provides faster
benchmark results and better equalizes benchmark testing
conditions. The benchmark conducts requests to each cloud in parallel,
so they experience similar network conditions. For example, in a
serial testing model when downloading data from a remote server, the
remote server may experience varying loads which will ultimately
result in different throughputs for the various tests. Our parallel
tests better equalize these conditions by having each cloud download
the data under the same network conditions.}

In the benchmark, we compute the means from 30 runs of a workflow that
includes one download data invocation, one train invocation, 30 upload
invocations, and 30 predict invocations. We run the workflows in
parallel on the separate clouds using multiprocessing on an eight-core
machine.

In \Figure{fig:7} we depict the combined runtime of our benchmark
tests. This allows us to compare the complete execution time of an AI
service workflow.

\OneFIGURE
    {ai-service-workflow-runtime.png}
    {Mean runtime of the Eigenfaces SVM workflow deployed
     as a multi-cloud service.}
    {fig:7}
    {1.0}
    
In \Table{tab:3} we provide complete test results for the multi-cloud benchmark.

\begin{comment}
\subsubsection{Pipelined Anova SVM Example}
\label{pipelined-anova-svm-example}
\end{comment}

\begin{comment}
\subsection{Caleb Example}\label{caleb-example}
\end{comment}

\COMMENT{
\section{Limitations}\label{limitations}

Azure has updated their libraries and discontinued the version 4.0 Azure
libraries. We updated \Cloudmesh to use the new library, but not all
features, such as virtual machine delete, are implemented or verified.
}


\section{Conclusion}
\label{sec:conclusion}

This paper has introduced a framework and tool called GAS Generator
that allows data scientists not experienced enough with REST and/or
OpenAPI to generate REST services from Python functions quickly. The
overall time for deploying the resulting service was reduced from
several months by inexperienced data scientists to about one hour
(excluding the time it takes to apply for cloud accounts). The
service can be provisioned on public clouds and shared with other
users. Authentication is built into our framework while leveraging
common REST service practices leveraging existing frameworks.
In a small benchmark executed on the
various cloud providers as well as local hardware, including Raspberry
PIs, we have seen that the cloud providers, when using similar
resources and images, perform similarly. To compare the services with
IoT devices such as Raspberry PI 3b+ and 4 we have chosen a small
enough example that can be conducted on them and can be used as a
reference to other IoT devices in the future. We found especially that
in the case of the PI 4, the performance was quite good for our
example. We also provided a cost-performance analysis to compare the
IoT devices with the cost used on the cloud to conduct the task over a
year's worth of activities. We find that the PI is very
cost-effective for this example.

However, our most significant gain from this project is the reduction
in manpower and entry barrier it takes to create and deploy our AI
services. Due to the generalized approach while using Python functions,
developers and data scientists can naturally integrate more complex
tasks as well as tasks that leverage cloud-specific AI services that
are uniquely offered by particular providers. GAS is an
open-source project, and we appreciate contributions to the
project. Please contact the first author
at \censor{\textit{laszewski\@gmail.com}}.


\section*{Acknowledgment}

\BLOCKCENSOR{A large section recognizing eight additional developers that were
involved in this effort besides the eight authors of this paper was removed to fulfull double
blind review conditions.}{We like to thank Brian Kegerreis, Jonathan
Beckford, Jagadeesh Kandimalla, Prateek Shaw, Ishan Mishra, Fugang
Wang, and Andrew Goldfarb for developing the Python REST service
generator on which this work is based on. We like to thank Caleb
Wilson for his contributions to this project. We also like to thank
the more than 70 contributors to the cloudmesh toolkit for developing
the various cloud providers. We would like to thank the NBDIF working
group for their contributions in discussions that lead to the
development of this effort.}



