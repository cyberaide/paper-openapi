% \documentclass{article}
%\documentclass[acmtog,review=false]{acmart}

\documentclass[conference]{IEEEtran}
\IEEEoverridecommandlockouts
% The preceding line is only needed to identify funding in the first footnote. If that is unneeded, please comment it out.



%\usepackage[disable]{todonotes}
\usepackage{todonotes}
\usepackage{fontawesome}
\usepackage{diagbox}
\usepackage{longtable}

\PassOptionsToPackage{hyphens}{url}
\usepackage[hidelinks]{hyperref}

\usepackage{booktabs,siunitx}
\usepackage{breakurl}

% \usepackage[table]{xcolor} 
\usepackage{fancyvrb}
\usepackage{listings}
\usepackage{xcolor}
\usepackage{tikz}
\usepackage{forest}
\usepackage{adjustbox}
\usetikzlibrary{arrows.meta,shadows}
\usepackage{array,multirow,graphicx}
\setcounter{tocdepth}{9}
\usepackage{enumitem} 

\usepackage{hyperref}

%\newcommand{\OK}{\faCheckSquare}
%\newcommand{\NO}{\faTimesCircle}
\newcommand{\OK}{\faCheck}
\newcommand{\NO}{\faTimes}

\newcommand{\TODO}[1]{\todo[inline]{#1}}



\newcommand{\TITLE}{A Survey of OpenAPI Tools for Service Generation with Python}


\begin{document}

\title{\TITLE
%\thanks{Identify applicable funding agency here. If none, delete this.}
}


\author{\IEEEauthorblockN{
Gregor von Laszewski\IEEEauthorrefmark{1}, 
Anthony Orlowski\IEEEauthorrefmark{1}, 
others add yourself once you contributed
}
\IEEEauthorblockA{\IEEEauthorrefmark{1}
\textit{Dept. of Intelligent Systems Engineering},
\textit{Indiana University},
Bloomington, IN 47408, USA \\
Email: laszewski@gmail.com} % 0000-0001-9558-179X (orcid)
}


\maketitle

\begin{abstract}

TBD

\end{abstract}

\begin{IEEEkeywords}
Software/Software Engineering, Tools,  Programming Environments/Construction
Tools,  Code generation, Distributed systems,
\end{IEEEkeywords}

{\bf\em Keywords---} {\em\bf code generation,
multi-cloud,
hybrid cloud,
OpenAPI,
REST}

\tableofcontents

\subsection{TODO}

\TODO{1 -- we need to verify date of last change in the bibtex entries that come form github and update the month and year accordingly. This way we can identify outdated or inactive projects easier}

\TODO{2 -- Create non plagiarized summary for each technology}

\TODO{3 -- find features unique to the technology, but do not do the advertisement trap. ABC is the best of (if the project claims this we can not say ... its a claim)}

\TODO{6 -- sort technologies into different sections}

\TODO{4 -- derive categories so we can do a table to compare them}

\TODO{5 -- cite Gregors cloud computing book with the rest and openapi chapters}

\TODO{7 -- prioritize the technology to be written about and do not spend much time on openapi 2.0. remember swagger (old) is openapi 2.0}

\TODO{11 - add citations to table}


\section{REST Service Generation Frameworks}

Categories:

We need categories and describe what they are

\begin{description}

\item[Language] -- REST is not dependent on a particular language, but many supporting libraries written in a particular language allow using REST services in a straight forward fashion. There are many REST APIs and frameworks available for the Python programming language. 
In this work we will focus on Python.


\item[Reviews and Collections] --

\item[Microservice] --

\item[Web Framework] --
A web framework is a software framework that is helping in the the development of web applications including web services. It integrates  web resources and web API's so users of the Framework can easily use and access them. Web frameworks promote development workflows for  simplifying the integration and development of  new services while reducing the overhead typically associated during the development phase for standard tasks. This may include security, database, Web page templates, session management, and code development and reuse.

\item[Client] -- Clients  deal with intaeracting easily with REST services. An efficient client will make the access easy and fast.

\item[Server] -- Servers are running in the background and can be scaled based on user demand. Most often servers are used from a Web Framework. 

\item[Code Generators] --

\item[Service Generators] -- 


\end{description}

\TODO{8 -- maybe we need database as category}

\subsection{Reviews and Collections}

15 OpenAPI Tools and frameworks \cite{www-python-rest-best}





\subsection{Reviews and Collections}

\begin{description}

\item[TODO]

\end{description}

\subsection{Microservice}

\begin{description}

\item[TODO]

\end{description}

\subsection{Web Framework}

\begin{description}

\item[TODO]

\end{description}

\subsection{Client}

\begin{description}

\item[TODO]

\end{description}

\subsection{Server}

\begin{description}

\item[tornado] todo \cite{www-tornado}
Description:
Features:

\item[Sanic] todo \cite{www-sanic}
Description:
Features:

\item[falcon] todo \cite{www-falcon}
Description:
Features:

\item[bottle] todo \cite{www-bottle}
Description:
Features:

\item[turbogears] todo \cite{www-turbogears}
Description:
Features:

\item[web2py] todo \cite{www-web2py}
Description:
Features:

\item[pyramid] todo \cite{www-pyramid}
Description:
Features:

\end{description}

\subsection{Service and Code Generators} 

Possible features:
\begin{itemize}

\item Supports multiple servers

\item Built in security (OAuth2, Simple)

\item Decoupled service and web-framework code (no function decorators or web framework code mixed with service logic)

\item OpenAPI specification generation version (2.0, 3.0, both)

\item Built-in interactive documentation

\item Routing (URI to function) (Explicit, Automatic)

\item Automatic parameter and response serialization and de-serialization, type casting, type validation

\item API specification generation (``API first'', ``Code first'')

\item API Versioning support

\end{itemize}

\begin{description}

\item[cloudmesh] \cite{cloudmesh-openapi} 

\item[Cloudmesh eve] \cite{www-cloudmesh-spec-eve}

\item[eve] \cite{www-eve} was a disbanded project for a while and recommended to use others we started with this and than switched to conexion swagger.

\end{description}


\subsection{Languages}


\begin{description}


\item[Flask RESTX] todo \cite{www-flaskrestx} 

\item[\href{https://flask-restful.readthedocs.io/}{Flask Restful}]
    \cite{www-flask-restful}

\item[\href{https://www.django-rest-framework.org/}{django}] \cite{www-django-rest}

\item[\href{https://github.com/zalando/connexion}{connexion}] An ``API first'' service framework that directs HTTP requests to Python functions based on a user-provided OpenAPI 3.0 specification. It provides automatic request routing, parameter and response serialization, optional strict type checking, supports multiple servers, and provides interactive documentation and API versioning. \cite{www-connexion}

\item[\href{https://requests.readthedocs.io/}{Requests}] \cite{www-python-requests}  just accessing

\item[\href{https://github.com/tiangolo/fastapi}{fastapi}] A service framework that generates an OpenAPI 3.0 specification from user code and routes HTTP requests accordingly. A user provides function decorators, such as ``@app.get(\"/items/\{item\_id\}\")'', to declare a functions path and HTTP method type. It uses Python type annotations to generate API specification and provide argument and response type checking and serialization, supports parameters from cookies, forms, headers, and files, and provides validation constraints. It supports multiple ASGI servers, interactive documentation, and authentication including OAuth 2 and HTTP Basic. \cite{www-fastapi}

\item[\href{https://github.com/hugapi/hug}{hugapi}] A service framework that generates an OpenAPI 3.0 specification from user service code extended with function decorators including HTTP methods and paths, and Python type annotations for function arguments. It supports multiple WSGI servers and provides built-in request testing and API versioning support. The framework provides user declared directives, input formatters, output formatters, middle-wear functions, and parameter mapping. \cite{www-hugapi}

\item[\href{https://github.com/rafaelcaricio/sticker}{sticker}] \cite{www-sticker} A service framework that supports multiple web frameworks. It enables ``pure python'' request handlers by requiring manual OpenAPI 3.0 specification and expecting handlers to return a specially formatted dictionary that is then parsed for the chosen web framework.

\item[\href{https://github.com/marshmallow-code/apispec}{apispec}] An OpenAPI 3.0 specification generator that is web-framework agnostic. It parses API specifications with doc-string utilities, and with built-in support for Marshmallow's Python object serialization. Multiple projects extend this project to tightly integrate it with a specific web-framework such as falcon or aiohttp. \cite{www-marshmallow} \cite{www-apispec}

\end{description}

\begin{table*}[htb]

\caption{Comparison of Python REST service frameworks}


\TODO{9 -- we may ave to flip the table}

\TODO{10 -- if a feature is missing we do not realy have to use \NO, we just leave it empty.}

\bigskip

\begin{tabular}{|l|l|l|l|l|}
\hline
& \multicolumn{4}{c|}{Features}\\
 Tools 
 & \rotatebox{90}{Reference~} 
 & \rotatebox{90}{Maintained~} 
 &  \rotatebox{90}{Web Service} 
 & \rotatebox{90}{Client} \\ \hline
Cloudmesh OpenAPI & \cite{cloudmesh-openapi}  & \OK & \OK &  \\ \hline
Cloudmesh Eve &  \cite{www-cloudmesh-spec-eve} & \NO & \OK &  \\ \hline
Connexion & \cite{www-connexion}  & \OK & \OK &  \\ \hline
Flask RESTX & \cite{www-flaskrestx} &  &  &  \\ \hline
Flask Restful & \cite{www-flask-restful} &  &  &  \\ \hline
Django REST & \cite{www-django-rest} &  &  &  \\ \hline
Requests & \cite{www-python-requests} &  &  & \OK \\ \hline
FastAPI & \cite{www-fastapi} &  &  & \OK \\ \hline
hugapi & \cite{www-hugapi} & \OK &  \OK &  \\ \hline
sticker & \cite{www-sticker}  & \OK & \OK &  \\ \hline
Falcon apispec & \cite{www-falcon-apispec} &  &  &  \\ \hline
Eve & \cite{www-eve}  &  &  &  \\ \hline
Django drf-spectacular & \cite{www-drf-spectacular} &  &  &  \\ \hline
Falcon apispec & \cite{www-falcon-apispec} &  &  &  \\ \hline
Pyramid OPenAPI 3 & \cite{www-pyramid-openapi3} &  &  &  \\ \hline
aio Openapi & \cite{www-aio-openapi} &  &  &  \\ \hline
aishtto-apispec & \cite{www-aiohttp-apispec} &  &  &  \\ \hline
Flask API Spec & \cite{www-flask-apispec} &  &  &  \\ \hline
Flask Swagger & \cite{www-flask-swagger} &  &  &  \\ \hline
Flask RestPlus & \cite{www-flask-restplus} &  &  &  \\ \hline
Flask Swagger ui & \cite{www-flask-swagger-ui} &  &  &  \\ \hline
OpenAPI core & \cite{www-openapi-core} &  &  &  \\ \hline
pyswagger & \cite{www-pyswagger} & \NO  &  &  \\ \hline
Django Rest Swagger & \cite{www-django-rest-swagger} & \NO &  &  \\ \hline
Django drf yasg &  \cite{www-django-rest-swagger}& \NO  &  &  \\ \hline
Tornado &  \cite{www-tornado} &  &  &  \\ \hline
Sanic & \cite{www-sanic} &  &  &  \\ \hline
Falcon &  \cite{www-falcon} &  &  &  \\ \hline
Bottle &  \cite{www-bottle} &  & \OK &  \\ \hline
Turbogears &  \cite{www-turbogears} &  & \OK &  \\ \hline
Web2py & \cite{www-web2py} &  & \OK &  \\ \hline
Pyramid &  \cite{www-pyramid} &  & \OK  &  \\ \hline
Flasgger &  \cite{www-flasgger} &  &  &  \\ \hline
SAFRS & \cite{www-safrs} &  &  &  \\ \hline
apispec & \cite{www-apispec} & \OK  &  &  \\ \hline
&  &  &  &  \\ \hline

\end{tabular}

\end{table*}

Two aspects
(a) focused on API frameworks for Python that use OpenAPI standard. 
(b) focus on API frameworks that focus on REST services


Within we see  (a) frameworks that rely on a ``specification first'' approach.  From the spec they can generate stub code in many languages. 
(b) frameworks to define ``code first'' and then generate the specification.

Anthony thinks these two that I thought were  similar to our efforts: fastapi, hugapi


These lists were helpful in quickly scanning for similar tools:

\begin {description}
\item[\href{https://swagger.io/tools/open-source/open-source-integrations/}{something swagger}] \cite{www-swagger-integration}

\item[\href{https://www.blazemeter.com/blog/how-to-generate-openapi-definitions-from-code}{some tutorial}] \cite{www-openapi-from-code}

\item[\href{https://openapi.tools/}{OpenAPI Tools}] \cite{www-openapi-tools}

\item[\href{https://apis.guru/awesome-openapi3/category.html}{APIs.guru}]\cite{www-apis-guru}

\item[\href{https://github.com/topics/openapi3}{GitHub Topics openapi3}]\cite{???}


\end{description}

There may be a combination of code and swagger tools for defining rest services as microservices. We need to evaluate if these have been improved,a s several years when we started this none of them rely did the functional approach we had and required openapi to be added in detail as part of the comments. This may have changed.

Frameworks that generate specifications from code but are limited to specific frameworks or extract specification provided in doc-strings or comments:

\begin{description}

\item[\href{https://github.com/thomaxxl/safrs}{SAFRS}] Generates specification from SQLAlchemy and Flask-Restful \cite{www-safrs}

\item[\href{https://github.com/flasgger/flasgger}{Flasgger}] Can use specification defined in comments of Flask views  \cite{www-flasgger}

\item[\href{https://github.com/axnsan12/drf-yasg}{drf-yasg}] Generates Swagger/OpenAPI 2.0 specification from Django Rest Framework API \cite{www-drf-yasg}

\item[\href{https://github.com/tfranzel/drf-spectacular}{drf-spectacular}] Generates OpenAPI 3.0 specification from Django Rest Framework API \cite{www-drf-spectacular}

\item[\href{https://github.com/alysivji/falcon-apispec}{falcon-apispec}] Generates Swagger/OpenAPI 2.0 specification from Falcon framework \cite{www-falcon-apispec}

\item[\href{https://github.com/Pylons/pyramid_openapi3}{pyramid\_openapi3}] Generates OpenAPI 3.0 specification from Pyramid framework \cite{www-pyramid-openapi3}

\item[\href{https://github.com/quantmind/aio-openapi}{aio-openapi}] Generates OpenAPI 3.0 specification with aiohttp framework \cite{www-aio-openapi}

\item[\href{https://github.com/maximdanilchenko/aiohttp-apispec}{aiohttp-apispec}] Extends apispec to be used with aiohttp \cite{www-aiohttp-apispec}

\item[\href{https://github.com/jmcarp/flask-apispec}{flask-apispec}] Extends apispec to be used with Flask \cite{www-flask-apispec}

\end{description}


\section{TODO}

These have not yet been added to bibtex

Flask-swagger for Swagger 2.0 only \cite{www-flask-swagger}

Flask-restplus \cite{www-flask-restplus,}

flask-swagger-ui \cite{www-flask-swagger-ui}

Unclear:

openapi-core \cite{www-openapi-core}

Disbandend:

pyswagger \cite{www-pyswagger} 

django-rest-swagger \cite{www-django-rest-swagger}


Support for OpenAPI is irrelevant:

 Another divide I've found in these tools is whether they support Swagger 2.0/OpenAPI 2.0 or Openapi 3.0. It seems many of the older ones for 2.0 are disbanded and recommend other frameworks for new projects.



For example:

drf-yasg is for 2.0, \cite{www-drf-yasg}

recommends below for 3.0

drf-spectacular \cite{www-drf-spectacular}

%
% References
%

\begin{table*}[htb]
\resizebox{1.0\textwidth}{!}{%
\begin{tabular}{p{2cm}p{2cm}p{2cm}p{2cm}p{2cm}p{4cm}p{2cm}}
Cloudmesh OpenAPI & converters & Python &  &  & Transform Python function specifications into OpenAPI services \\
openapi-filter & parsers & Node.js & True &  & OpenAPI 2.0 and 3.0 filter utility. A CLI/module to filter out internal/private paths, operations, parameters, schemas etc from OpenAPI v1/OpenAPI v2/AsyncAPI definitions. Simply flag any OpenAPI object within the definition with an `x-internal` specification extension or target a OpenAPI property (tags, methods, OperationId), and it will be removed from the output. \\
oas-gen & ['sdk', 'server'] & Java, TypeScript &  &  & An engine to generate clients and server stubs in different languages by OpenAPI. It's simple and extendable by design. Currently contains generators for Srping Web, Reactor Netty and JavaScript standard library. \\
Goa & dsl & Go & True &  & Goa provides a holistic approach for developing remote APIs and microservices in Go. implementers don't have to worry about the documentation getting out of sync as Goa takes care of generating OpenAPI specifications for HTTP based services and gRPC protocol buffer files for gRPC based services \\
Rusty API Modeler & code-generators & Rust &  &  & Language-agnostic openapi code generator. Requires developer to write handlebars template files for API generation. Comes with a few built-in templates for testing and demonstration purposes in different languages.
 \\
openapi-filter & parsers & Node.js & True &  & OpenAPI 2.0 and 3.0 filter utility. A CLI/module to filter out internal/private paths, operations, parameters, schemas etc from OpenAPI v1/OpenAPI v2/AsyncAPI definitions. Simply flag any OpenAPI object within the definition with an `x-internal` specification extension or target a OpenAPI property (tags, methods, OperationId), and it will be removed from the output. \\
CUE & dsl & CUE &  &  & CUE is an open source language, with a rich set of APIs and tooling, for defining, generating, and validating all kinds of data configuration, APIs, database schemas, code, etc. CUE currently supports generating OpenAPI through its API. \\
Supermodel & dsl & SaaS & True &  & Model your data using JSON Schema, refer and remix the models freely, convert to various formats including OAS v2/v3. \\
Apimatic Transformer & converters & SaaS & True &  & Transform API Descriptions to and from RAML, API Blueprint, OAI v2/v3, WSDL, etc. \\
Google Gnostic & converters & Go & True &  & Compile OpenAPI descriptions into equivalent Protocol Buffer representations \\
swagger2openapi & converters & Node.js / CLI & True &  & Upgrade files from OpenAPI v2.0 to v3.0, bundling into one mega file or respecting \$refs. Part of oas-kit. \\
OAS RAML Converter & converters & Node.js & True &  & Converts between OpenAPI and RAML API specifications \\
OData OpenAPI & converters & XSLT & True &  & OData 4.0 to OpenAPI v3.0 converter \\
OpenAPI Filter & converters & Node.js & True &  & Filter internal components from OpenAPI Descriptions \\
OData.OpenAPI & converters & .NET &  &  & Convert an Edm (Entity Data Model) to OpenAPI 3.0 \\
pyswagger & converters & Python & True &  & Client & converter in Python, which is type-safe, dynamic, spec-compliant. \\
odata2openapi & converters & Node.js & True &  & OData 4.0 to OpenAPI v2.0 converter \\
avantation & converters & TypeScript &  &  & Generate OpenAPI 3.x specification from HAR. \\
OpenDocumenter & documentation & Vue.js & True &  & OpenDocumenter is a automatic documentation generator for OpenAPI v3 schemas. Simply provide your schema file in JSON or YAML, then sit back and enjoy the documentation. \\
ReadMe & documentation & SaaS & True &  & Build beautiful, personalized, interactive developer hubs \\
APIMatic Developer Experience Portal & documentation & SaaS & True &  & Customizable developer portals packed with language specific documentation, client libraries, code samples, an API console and much more. \\
APITree & documentation & SaaS & True &  & HUB for managing and sharing APIs. Converts OpenAPI v2 / v3 files into beautiful API documentation. \\
Kong Enterprise Edition & documentation & Lua & True &  & Highly customizable developer portal with developer onboarding, integrated with the Kong API Gateway \\
RestCase Docs & documentation & SaaS & True &  & An API-first and security-first management platform. Design visually and we will create a beautiful API documentation for your APIs. \\
LucyBot DocGen & documentation & JavaScript & True &  & Generate a customizable website, with API documentation, console, and interactive workflows, from an OpenAPI spec \\
Nexmo OAS Renderer & documentation & Ruby &  &  & Ruby OpenAPI docs rendering, use standalone or add to your Rails app \\
openapi-viewer & documentation & Vue.js &  &  & Browse and test a REST API described with the OpenAPI 3.0 Specification \\
ReDoc & documentation & React.js & True &  & OpenAPI-generated API Reference Documentation \\
oas3-api-snippet-enricher & documentation & JavaScript &  &  & Enrich your existing description documents with generated code samples \\
widdershins & documentation & Node.js & True &  & Generate Slate/Shins markdown from OpenAPI 2.0/3.0.x \\
MrinDoc & documentation & Vue.JS & True &  & OpenAPI description document viewer. \\
RapiDoc & documentation & Custom Element & True &  & Custom Element to view OpenAPI descriptions. \\
RapiPdf & documentation & Custom Element & True &  & Custom Element to generate PDF from OpenAPI descriptions. \\
Stoplight Docs & documentation & SaaS & True &  & Create beautiful, customizable, interactive API documentation generated from OpenAPI, integrated with Stoplight Studio \\
jekyll-openapi & documentation & Jekyll &  &  & An OpenAPI 3 documentation website generator built with Jekyll for use on GitHub Pages. \\
BOATS & dsl & Node.js & True &  & BOATS allows for larger teams to contribute to multi-file OpenAPI definitions by writing Nunjucks tpl syntax in yaml with a few
important helpers to ensure stricter consistency, eg operationId: <$ uniqueOpId() $>.
 \\
Spot & dsl & TypeScript & True &  & A concise, developer-friendly way to describe your API contract. \\
generator-openapi-repo & code-generators & JavaScript & True &  & Generate the repository structure for a scalable OpenAPI Description \\
OpenAPI Server Code Generator (oapi-codegen) & ['code-generators', 'miscellaneous'] & Go &  &  & Generate a web service using the [Echo](https://github.com/labstack/echo) framework from an OpenAPI v3 specification \\
OpenAPI Client Generators & ['code-generators', 'sdk'] & C# & True &  & .NET Core command line program to generate strongly typed client API codes in C# on .NET Frameworks and .NET Core, and in TypeScript for Angular 5+, Aurelia, jQuery, AXIOS and Fetch API. \\
OpenAPI Generator & ['code-generators', 'sdk'] & Java & True &  & A template-driven engine to generate documentation, API clients and server stubs in different languages by parsing your OpenAPI Description (community-driven fork of swagger-codegen) \\
Bump & documentation & SaaS & True &  & Bump generates elegant documentation and changelogs from your OpenAPI specifications. Git diff, for your API. \\
Python OpenAPI Generator & code-generators & Python &  &  & This library facilitates creating OpenAPI document for Python projects. \\
Apitive Studio & documentation & Angular 7.0, Java / Saas & True &  & A platform for Digital Product Managers and API Consultants to design REST APIs with in-built mock and documentation.
 \\
VSCode OpenAPI & text-editors & Any & True &  & OpenAPI extension for Visual Studio Code - new file templates, navigation, intellisense, code snippets. \\
KaiZen-OpenAPI-Editor & text-editors & Java & True &  & Full-featured Eclipse editor for OpenAPI, also available on Eclipse Marketplace. \\
Atom/linter-swagger & text-editors & JavaScript & True &  & This plugin for Atom Linter will lint OpenAPI, both JSON and YAML using swagger-parser node package. \\
Atom/linter-openapi & text-editors & JavaScript &  &  & This plugin for Atom Linter will lint OpenAPI YAML files using openapi-enforcer node package. \\
Swagger Editor & text-editors & Node.js & True &  & Design, describe, and document your API on the first open source editor fully dedicated to OpenAPI-based APIs. \\
SwaggerHub & text-editors & SaaS/On-Premise NodeJS & True &  & API design and documentation platform to improve collaboration, standardize development workflow and centralize their API discovery and consumption. \\
Senya Editor & text-editors & Java & True &  & JetBrains IDE plugin to show Swagger UI as a preview, for visual feedback as you edit. \\
VSCode/openapi-lint & text-editors & Node.js & True &  & OpenAPI 2.0/3.0.x intellisense, validator and linter for Visual Studio Code \\
RepreZen API Studio & gui-editors & Java & True &  & RepreZen API Studio is an integrated workbench that brings API-first design into focus for your whole team, harmonizes your API designs, and
generates APIs that click into client apps.
 \\
Stoplight Studio & ['gui-editors', 'text-editor'] & Desktop / SaaS & True &  & Stoplight Studio is a GUI/text editor with linting and mocking built right in.  It can run on the desktop with local files, and in the browser powered by your  existing GitHub, GitLab, or BitBucket repos.
 \\
Hackolade & gui-editors & ReactJS & True &  & A visual editor for OpenAPI v2/v3, from the pioneer in data modeling for NoSQL databases.
 \\
Apitive Studio & gui-editors & Angular 7.0, Java / Saas & True &  & A platform for Digital Product Managers and API Consultants to design REST APIs with in-built mock and documentation.
 \\
Apicurio Studio & gui-editors & Angular 7.0, Java / Saas & True &  & Web-Based Open Source API Design via the OpenAPI specification.
 \\
OAIE Sketch & gui-editors & Vue.js &  &  & Browser based OpenApi Integrated Editor with side-by side view of the yaml and an interactive graph.
 \\
ApiBldr & gui-editors & Angular 9.0 / Saas & True &  & Web-Based API Designer for OpenAPI (swagger) and AsyncAPI specifications.
 \\
RestCase Designer & gui-editors & Angular 9.0 / Saas & True &  & A design-first API managment platform with WYSIWYG API Designer for OpenAPI and AsyncAPI specifications.
 \\
Optic & learning & cli &  &  & A proxy server that watches an API and helps you build an OpenAPI description interactively. \\
Meeshkan & ['learning', 'mock'] & Python &  &  & Mock HTTP APIs through a combination of API definitions, recorded traffic and code. Used for sandboxes, as well as automated and exploratory testing. \\
Swagger Inspector & learning & SaaS & True &  & Run mock requests in a webapp and Swagger Inspector infers your OpenAPI description. \\
Response2Schema & learning & PHP &  &  & Takes any JSON response and generates an OpenAPI definition document with the component schema and a default endpoint. \\
InducOapi & learning & Python &  &  & A simple python module to generate OpenAPI Description Documents by supplying request/response bodies. \\
Connexion & mock & Python & True &  & OpenAPI First framework for Python on top of Flask with automatic endpoint validation & OAuth2 support \\
Prism & mock & cli & True &  & Turn any OAI file into an API server with mocking, transformations, validations, and more. \\
Sandbox & mock & SaaS / Java & True &  & SaaS, self-hosted, or CLI tool for turning OpenAPI (and other) descriptions into a mock server, where you can modify behaviour, simulate downtime, and any other nonsense you can think of thanks to a built-in code editor! \\
Microcks & mock & Self-hosted / SaaS & True &  & Mocking and testing platform for API and microservices. Turn your OAI contract examples into ready to use mocks. Use examples to test and validate implementations according schema elements. \\
API Sprout & mock & cli / Docker &  &  & Lightweight, blazing fast, cross-platform OpenAPI 3 mock server with validation \\
OpenAPI Mocker & mock & nodejs &  &  & Standalone nodejs based OpenAPI 3 mock server, docker-friendly with request validation and autoreload. \\
MockLab & mock & SaaS & True &  & SaaS platform to upload your spec to create a mock server \\
Fakeit & mock & cli / Docker &  &  & Create mock server from OpenAPI 3 specification with random response generation and request validation. \\
Unmock & ['mock', 'testing'] & Node.js &  &  & API integration testing library that intercepts outgoing requests and serves back mock data based on the OpenAPI descriptions. \\
Apitive Studio & mock & Angular 7.0, Java / Saas & True &  & A platform for Digital Product Managers and API Consultants to design REST APIs with in-built mock and documentation.
 \\
tsoa & ['server', 'data-validation'] & TypeScript & True &  & Creates OpenAPI docs and provides free runtime validation for your Koa, Express, Hapi (and more) services \\
Vert.x Web Api Contract & server & Java, Kotlin, JavaScript, Groovy, Ruby, Ceylon & Scala &  &  & Create API endpoints with Vert.x 3 and OpenAPI 3 with automatic requests validation \\
Vert.x Web API Service & server & Java &  &  & Create API service proxies using event bus with request/response validation \\
BaucisJS + baucis-openapi3 & server & JavaScript &  &  & Create REST resources with persistence on MongoDB and expose OpenAPI v.3 contracts \\
BaucisJS + baucis-swagger2 & server & JavaScript & True &  & Create REST resources with persistence on MongoDB and expose OpenAPI v.2 contracts \\
@smartrecruiters/openapi-first & server & Node.js &  &  & Initializes your API express application with the description in OpenAPI 3.0 format using
provided middlewares (parsers, validators, controller, defaults setting) or custom ones
 \\
openapi-backend & server & Node.js &  &  & Build, Validate, Route, and Mock using OpenAPI specification. Framework-agnostic \\
OpenAPI Enforcer Middleware & server & Node.js & True &  & An express middleware that makes it easy to write web services that follow an OpenAPI specification by leveraging the tools provided in the openapi-enforcer package. \\
MicroTS & server & Node.js & True &  & Take an OpenAPI description and generate TypeScript projects via Docker. \\
API Platform & server & PHP & True &  & REST and GraphQL framework to build modern API-driven projects \\
Mojolicious::Plugin::OpenApi & server & Perl & True &  & Mojolicious::Plugin::OpenAPI is a plugin for Mojolicious framework that add routes and input/output validation to your Mojolicious application based on OpenAPI description documents.'
 \\
Fusio & server & PHP &  &  & Open source API management platform \\
yii2-app-api & ['server', 'mock'] & PHP &  &  & Generate Server side API code with routing, models, data validation and database schema from an OpenAPI description. Based on Yii Framework.
 \\
@eropple/nestjs-openapi3 & ['server'] & Node.js &  &  & Integrates tightly with a NestJS application to infers complex descriptions and expresses them in its generated OpenAPI document. It then presents that document via ReDoc, and validates inputs for conformance to spec.
 \\
@nestjs/swagger & ['server'] & TypeScript &  &  & Official OpenAPI (Swagger) module for NestJS. Use decorators to define OpenAPI endpoint documentation, parameters and return types. Integrates tightly with a NestJS application. Ships with Swagger UI and serves OpenAPI v3 spec.
 \\
Falcon Heavy & ['server', 'mock'] & Python &  &  & The framework for building app backends and microservices via the API design-first workflow.
 \\
BigstickCarpet/swagger-cli & description-validators & Node.js / CLI & True &  & Simple validation for OpenAPI files, supporting JSON/YAML and v2/v3 description documents. \\
openapi-spec-validator & description-validators & Python & True &  & OpenAPI Description validator \\
Dredd & ['testing'] & Javascript & True &  & Language-agnostic command-line tool for validating API description document against backend implementation of the API \\
express-openapi-validator & ['miscellaneous', 'description-validators', 'data-validators'] & JavaScript &  &  & �� Auto-validate API requests and responses in ExpressJS. \\
openapi-dev-tool & ['miscellaneous', 'testing', 'documentation'] & JavaScript &  &  & OpenAPI Dev Tool proposes to developers a unique tool to address development and industrialization needs! \\
openapi-spring-webflux-validator & ['miscellaneous', 'description-validators', 'data-validators'] & Java/Kotlin & True &  & �� A friendly kotlin library to validate API endpoints using an OpenAPI 3.0 or OpenAPI 2.0 specification \\
Spectral & description-validators & CLI & TypeScript/JavaScript & True &  & A flexible JSON/YAML object linter with portable "rulesets" and custom functions. \\
OpenAPI Style Validator & description-validators & Java, CLI & True &  & A customizable style validator to make sure your OpenAPI description follows your organization's standards. \\
OpenAPI Validator & description-validators & Node.js & True &  & Configurable and extensible validator/linter for OpenAPI documents \\
express-ajv-swagger-validation & data-validators & Node.js & True &  & Express middleware which validates request body, headers, path parameters and query parameters according to an OpenAPI Description \\
committee & data-validators & Ruby & True &  & Validation middleware for Rack server. This gem validates request and response using an OpenAPI Description. And convert parameter string to specific Ruby object (e.g. convert datetime string to DateTime class). \\
OpenAPI Enforcer & ['data-validators', 'description-validators', 'testing'] & Node.js & True &  & Validate your OpenAPI document, serialize, deserialize, and validate incoming requests and outgoing responses, and simplify response building. You can even produce mock data. \\
openapi4j & ['data-validators', 'schema-validators', 'parsers'] & Java &  &  & Parse Description Document, validate API requests and responses using OpenAPI 3.x. \\
oas-tools & ['security', 'documentation', 'server', 'parsers', 'data-validators', 'description-validators'] & Node.js &  &  & NodeJS module to manage RESTful APIs defined with OpenAPI 3.0 Description over express servers, including security validations \\
openVALIDATION & ['miscellaneous', 'description-validators', 'data-validators'] & Java &  &  & Allows complex validation rules to be specified in openAPI spec files using natural language. \\
swagger-parser & parsers & Java & True &  & Swagger Parser reads OpenAPI definitions into current Java POJOs. \\
BigstickCarpet/swagger-parser & parsers & Node.js & True &  & Swagger/OpenAPI 2.0 and 3.0 parser and validator. Can also bundle multiple files into one via `$ref`. \\
KaiZen OpenAPI Parser & parsers & Java &  &  & High-performance Parser, Validator, and Java Object Model for OpenAPI 3.x \\
OpenAPI-TS & parsers & TypeScript &  &  & TS Model & utils for OpenAPI 3.0.x contracts \\
kin-openapi & ['parsers', 'data-validators'] & Go &  &  & A Go library for handling OpenAPI 3.0 specifications \\
openapi-psr7-validator & data-validators & PHP &  &  & Using a PHP framework that supports PSR-7? Get free validation without writing a bunch of code, by registering this middleware and pointing it at your API description document. \\
php-openapi & ['parsers', 'description-validators'] & PHP &  &  & A PHP library for manipulating and validating OpenAPI 3.0 Descriptions \\
Object Oriented OpenAPI Specification & ['parsers'] & PHP &  &  & An object oriented approach to generating OpenAPI Descriptions, implemented in PHP \\
OpenAPI3-Rust & parsers & Rust &  &  & Rust serialization library for OpenAPI v3 \\
psx-api & parsers & PHP & True &  & Parse and generate API specification formats \\
Microsoft/OpenAPI.NET & parsers & .NET & True &  & C# based parser with OpenAPI Description validation and migration support from V2 \\
oas_parser & parsers & Ruby &  &  & A Ruby parser for OpenAPI 3.0+ descriptions. \\
openapi3 & parsers & Python &  &  & An OpenAPI 3 Specification client, and validator, covering both description validation and limited data validation for Python 3. \\
openapi3_parser & parsers & Ruby &  &  & A Ruby implementation of parser and validator for the OpenAPI 3 Specification. \\
APIMatic CodeGen & sdk & SaaS & True &  & Bring in your API description (OAI v2/v3, RAML, API Blueprint, WSDL, etc.) to generate fully functional SDKs in over 10 languages. \\
janephp/open-api & sdk & PHP & True &  & Generate a PHP Client API (PSR-7 compatible) given a OpenAPI specification. \\
go-swagger & ['parser', 'sdk', 'converters'] & Go & True &  & Unmaintained v2.0 only project seeking new maintainer, or probably a fork. Parser, validator, generates descriptions from code, or code from descriptions! \\
guardrail & ['sdk'] & Scala, Java, ... & True &  & Principled code generation from OpenAPI descriptions \\
restful-react & ['sdk'] & React (Typescript) & True &  & Generate React hooks with appropriate type-signatures from OpenAPI descriptions \\
NSwag &  & .NET & True &  & OpenAPI toolchain for .NET, Web API and TypeScript \\
api-codegen-ts &  & TypeScript & True &  & Generates TypeScript models, response validators, and operation controllers from OpenAPI descriptions \\
Swagger-Codegen & code-generators & Java & True &  & Swagger Codegen enables generating server stubs and client SDKs for APIs described in OpenAPI \\
Azure AutoRest &  & TypeScript & True &  & Generates client libraries for accessing RESTful web services from an OpenAPI document. Supports C#, PowerShell, Go, Java, Node.js, TypeScript, Python, and Ruby. \\
spring-openapi &  & Java &  &  & OpenAPI v3 generator for Java Spring. Includes also client generation. Supports inheritance with discriminators, Jackson annotations and custom interceptors. \\
openapi-diff & miscellaneous & Java &  &  & Utility for comparing two OpenAPI specifications. \\
$ oas (CLI) & miscellaneous & JavaScript & True &  & Generate OAS files from code comments and easily host them ($ npm install oas -g) \\
openapi-cli-tool & miscellaneous & Python &  &  & Can list up defined API paths and bundle multi-file into one. Supports multiple file extensions. \\
laravel-openapi & ['converters', 'miscellaneous'] & PHP &  &  & Generate OpenAPI 3 specification for Laravel Applications. \\
Flotiq - headless CMS with OpenAPI support & ['sdk', 'gui-editors', 'miscellaneous'] &  &  &  & Visually define your Content Types, Flotiq automatically generates your own OpenAPI v3 compatible endpoints, SDKs and Postman collections. \\
Chai OpenAPI Response Validator & testing & Node.js & True &  & Simple Chai support for asserting that HTTP responses satisfy an OpenAPI spec. \\
jest-openapi & testing & Node.js & True &  & Additional Jest matchers for asserting that HTTP responses satisfy an OpenAPI spec. \\
hikaku & testing & Kotlin &  &  & A library that tests if the implementation of a REST-API meets its specification. \\
Swagger Inspector & testing & Self-hosted/SaaS & True &  & Swagger Inspector is a free online tool to quickly execute any API request, validate its responses and generate a corresponding OpenAPI Description. \\
Assertible & testing & SaaS & True &  & Import an OpenAPI specification into Assertible to generate tests that validate JSON Schema responses and status codes on every endpoint. \\
Tcases for OpenAPI & testing & Java &  &  & Generates test cases directly from an OpenAPI v3 description of your API. Creates tests executable using various test frameworks. Bonus: Semantic linter reports elements that are inconsistent, superfluous, or dubious. \\
Schemathesis & testing & Python & True &  & Reads the description document and generates test cases that will ensure that your application is compliant with its description. \\
EvoMaster & testing & Java/Kotlin & True &  & A tool for automatically generating system-level test cases for RESTful APIs, using Evolutionary Algorithms and Dynamic Program Analysis. \\
StackHawk HawkScan & security & SaaS & True &  & StackHawk is an application vulnerability scanner purpose built for developers to use in the DevOps pipeline. It leverages a provided OpenAPI v2 or v3 spec file for route discovery and enhanced scanning. \\
42crunch & security & SaaS & True &  & A unique set of integrated API security tools that allow discovery, remediation of OpenAPI vulnerabilities and runtime protection against API attacks. \\
API Contract Security Audit & security & SaaS & True &  & Upload OpenAPI file, get a detailed report with located vulnerabilities, possible attack scenarios, ways to remediate. \\
OpenAPI Schema to JSON Schema & converters & JavaScript &  &  & Due to the OpenAPI v3.0 and JSON Schema discrepancy, you can use this JS library to convert OpenAPI Schema objects to proper JSON Schema. \\
JSON Schema to OpenAPI Schema & converters & JavaScript &  &  & Due to the OpenAPI v3.0 and JSON Schema discrepancy, you can use this JS library to convert JSON Schema objects to OpenAPI Schema. \\
Unchase.OpenAPI.Connectedservice & ['sdk', 'code-generators'] & .NET & True &  & Visual Studio extension to generate C# (TypeScript) HttpClient (or C# Controllers) code for OpenAPI web service with NSwag. \\
oazapfts! & sdk & TypeScript & True &  & Generate TypeScript clients from a given OpenAPI description document. \\
openapi-processor-spring & ['server'] & Java &  &  & Generates java interfaces & model classes for Spring Boot (annotation based, MVC & WebFlux) from an openapi.yaml. Provides type mapping capabilities to adjust the generated code. Gradle support. \\
har2openapi & converters & TypeScript &  &  & Automatically generate OpenAPI 3.0 Spec by using network requests captured in one or more HAR files \\
oa-client & ['miscellaneous', 'sdk'] & TypeScript &  &  & Flexible client helper for making and validating calls to OpenAPI backends. For Node and the browser. Runtime lib - no need for code generation! \\
Restish & ['documentation', 'miscellaneous', 'testing'] & CLI / Go &  &  & A CLI for REST-ish APIs with HTTP/2, built-in auth, content negotiation, caching, and more that understands and can discover OpenAPI descriptions. \\
openapi-examples-validator & ['miscellaneous', 'description-validators', 'data-validators'] & JavaScript & True &  & Validates embedded JSON-examples in OpenAPI-specs \\
openapi-to-postman & converters & JavaScript &  &  & Convert OpenAPI 3.0 specs to the Postman Collection (v2) format \\
super-linter & description-validators & CLI / Docker & True &  & GitHub Action to lint repositories as part of CI/CD. Implements the latest version of Spectral. \\
SpringFox & server & Java, Kotlin, Groovy, Ruby & True &  & Automated JSON API documentation for API's built with Spring and SpringBoot \\
php-openapi-faker & miscellaneous & PHP &  &  & Library to generate fake data for OpenAPI 3.x requests, responses and schemas. \\
OWASP ZAP & security & Java & True &  & OWASP ZAP is a free and open source web security tool that can be used manually or completely automated. It supports importing OpenAPI v2 and v3 definitions to allow an API to be thoroughly security tested. \\
JSON Designer & gui-editors & iOS/Swift &  &  & Visualize JSON models from imported OpenAPI YAML. Edit models and export OpenAPI YAML. \\
openapi-core & data-validators & Python &  &  & Validate your requests and responses against an OpenAPI 3 specification and get very verbose and human-readable descriptions of errors. You will receive a deserialized object along with validation result, so you won't need to deserialize it twice. \\
LoopBack 4 & server & TypeScript &  &  & A highly extensible object-oriented Node.js and TypeScript framework for building APIs and microservices with tight OpenAPI 3 integration. Serves Swagger UI and OpenAPI 3 spec out of the box. Generate code to interact with other OpenAPI-compliant APIs, or generate new API endpoints based on existing OpenAPI specs.
 \\
OpenAPI3 Fuzzer & data-validators & Python &  &  & Simple fuzzer for OpenAPI 3 specification based APIs. Verifies responses and sends various attack patterns. \\
vREST NG & testing & JavaScript & True &  & vREST NG is a simple and powerful application for API Automation. It Allows to use OpenAPI specification into vREST NG to drive your API testing that validates the API responses against JSON Schema and also provides powerful response validation capabilities. \\
openapi-validator-bundle & data-validators & PHP &  &  & Validates Request and Response using Symfony Framework \\
OpenAPI HttpFoundation Testing & data-validators & PHP &  &  & Strengthen your API tests by validating HttpFoundation responses against OpenAPI definitions \\
VSCode OpenAPI Snippets & text-editors & Any &  &  & OpenAPI Snippets for Visual Studio Code editor, includes split file validation \\
Spectator & testing & PHP &  &  & Spectator provides light-weight OpenAPI testing tools you can use within your existing Laravel test suite. \\
APIFuzzer & data-validators & Python & True &  & Fuzz test your application using your OpenAPI definition without coding. Integrate into CI/CD, get Junit XML test result and JSON report of failures \\
\end{tabular}
}
\end{table*}


%\bibliographystyle{ACM-Reference-Format}
\bibliographystyle{IEEEtran}
\bibliography{bib/references}



\end{document}